\chapter{Requirement Analysis}

In this chapter we discuss the requirements for our proxy. Since we're creating
a proxy aimed at use in a military context, military operational requirements
are part of the requirements for this proxy. As discussed earlier this proxy
must provide robustness and reliable delivery of data in \gls{dil} environments.

The first requirements are the premises we discussed in the introduction. The
proxy should be able to receive and forward HTTP-requests.

Next, the proxy must be able to handle the difficult network conditions of
dil. It must handle very low data rates and handle brief disconnects. It
should also support disconnects over a longer period of time by having store-
and-forward capabilities.

Payload agnostic. The data type doesn't matter(JSON, XML etc.).

In order to perform compression the proxy must be able to modify the payload
of the message. Due to security mechanisms that detect changes to the
payload(checksums), the payload must be restored back to its original form
before being forwarded to the final receiver.

As discussed in \cref{section:DIL-problems}, the dependency on end-to-end
connections needs to be removed. This can be done by adding proxies to the
network. Mobile units have to carry batteries with them and the capacity is
therefore limited. Advanced compression techniques may reduce the overhead,
but also requires more battery. This trade-off needs to be considered.


\begin{table}[h]
\begin{tabular}{| l | l |}
\hline
  \textbf{Requirement} & \textbf{Priority} \\ \hline
  Receive and forward HTTP requests & 1\\ \hline
  Allow modifications on the payload & 1 \\ \hline
  Retain the checksum of the payload & 1 \\ \hline
  Allow configuration of HTTP timeouts & 1 \\ \hline
  Keep HTTP-connection alive & 1 \\ \hline
  Support protocol X and y & 2 \\ \hline
  Handle very low data rate & 1 \\ \hline
  Have store-and-forward capabilities & 1 \\ \hline
  Handle frequent disconnects & 1\\ \hline
\end{tabular}
\caption{Summary of proxy requirements}
\end{table}


\section{Optimization techniques}

As we saw in the previous chapter there has been quite amount of research into
optimization techniques in \gls{dil} environments. In this section summarize
which techniques we want to support in our proxy.


%%%%%%%% TABLE OPTIMALIZATION OVERVIEW %%%%%%%%%%%%%
\begin{table}[h]
\begin{tabularx}{\textwidth}{| X | X |}
\hline
  \textbf{Protocol Stack} & \textbf{optimization possibilities} \\ \hline
  The application & Optimize the application\\ \hline
  Web service messaging: SOAP & Optimize SOAP, e.g XML compression \\ \hline
  HTTP/TCP, UDP or other transport protocols & SOAP is transport agnostic. Other
  protocols can be used. \\ \hline
  IP & NATO NEC feasibility study states that all protocols should be over IP. \\
  \hline
  Lower layers & Not in the scope of this thesis. \\ \hline
\end{tabularx}
\caption{Optimization possibilities.} \label{table:optimalization-overview}
\end{table}


\subsection{Compressing the payload}

The first optimization techniques deals with the optimization of the encoding.
By compressing the Web service payload, we can reduce the amount of data that
need to be sent. This optimization technique addresses one of the many
challenges of tactical networks, namely the bandwidth consumption due to large
message sizes.

\begin{itemize}
\item GZIP

\item EFX(Efficient XML). XML spesifikt, får ikke brukt på REST når vi har andre
payloads..

\end{itemize}

\subsection{Reducing overhead of SOAP}

HTTP/TCP is the most used transport protocol for SOAP messages, but since SOAP
is transport protocol agnostic different protocols can be used. Experiments show
that this is possible.


\section{Summary}

In this chapter we have discussed the requirements for our proxy. See table xx
for a summary. Next we discuss the design and implementation of our proxy.
