\chapter{Requirement Analysis}

In this chapter we discuss the requirements for the proxy being developed as a
part of this thesis. Those requirements builds on the scope and premises
discussed in the introduction. To recap, these defining requirements were:

\begin{enumerate}
    \item Support both REST and Web service communication between machines connected in a DIL network.
    \item Work on top of the IP-layer.
    \item All optimization techniques must be placed in a proxy, and not in the Web service applications themselves.
\end{enumerate}

\section{A DIL HTTP Proxy}

The first requirement implies that our proxy must accept HTTP, as this is the
far most used Web service protocol. Our proxy must be able to accept a HTTP
requests from a Web service, forward it to the other proxy, which in turn
delivers it to the intended receiver. The communication between the proxies are
not required to be HTTP, but rather a protocol than deals with DIL networks in a
better way. However, since ultimately a HTTP request should be delivered to the
intended receiver, the HTTP properties must be retained. This means that the
proxy must preserve the HTTP Method and HTTP headers. Also, since REST is
payload agnostic, the proxy muse be able to support different types of data
being sent through it(XML, JSON etc.).

Furthermore, the proxy must be able to handle the difficult network conditions
of DIL. It must handle very low data rates and handle brief disconnects. It
should also support disconnects over a longer period of time by having store-
and-forward capabilities.

\section{Support optimization techniques}

\subsection{Compression}

In order to perform compression the proxy must be able to modify the payload of
the message. Due to security mechanisms that detect changes to the
payload(checksums), the payload must be restored back to its original form
before being forwarded to the final receiver.

\subsection{Proxy protocol communication}

One of the optimization techniques identified was the usage of alternative
protocols. The proxy should therefor support a sub set of them, and be easily
configured to use other protocols for testing purposes.

\section{Other considerations}

Since we're creating a proxy aimed for use in a military context, military
operational requirements are part of the requirements for this proxy. Mobile
units have to carry batteries with them and the capacity is therefore limited.
Advanced compression techniques may reduce the overhead, but also requires more
battery. This trade-off needs to be considered.



\section{Summary}

In this chapter we have discussed the requirements for our proxy, which are summarized in \cref{table-proxy-requirements}. Next we discuss the design and implementation of our proxy.


\begin{table}[h]
\begin{tabular}{| l | l |}
\hline
  \textbf{Requirement} & \textbf{Priority} \\ \hline
  Receive and forward HTTP requests & 1\\ \hline
  Allow modifications on the payload & 1 \\ \hline
  Retain the checksum of the payload & 1 \\ \hline
  Allow configuration of HTTP timeouts & 1 \\ \hline
  Keep HTTP-connection alive & 1 \\ \hline
  Support protocol X and y & 2 \\ \hline
  Handle very low data rate & 1 \\ \hline
  Have store-and-forward capabilities & 1 \\ \hline
  Handle frequent disconnects & 1\\ \hline
\end{tabular}
\caption{Summary of proxy requirements}
\label{table-proxy-requirements}
\end{table}
