\chapter{Technical Background}
\label{chapter:background}

In this chapter we present the technical background of the central concepts and
protocols this thesis is based on. We first give an introduction to computer
networks in general and how they are organized. Next, we introduce a set of
network metrics used to characterize different types of DIL networks in this
thesis. Then we look into two very common communication patterns. Next, we
present the W3C Web service technology commonly used for exchanging data in
military systems. We also introduce the REST style of services. Finally, we look
into a number of protocols that we can replace HTTP/TCP with in order to
increase the performance of Web services.


\section{Computer Networks}

A computer in a network is often referred to as a \textit{node}. These nodes
can be interconnected and form large computer networks. The most well-known
network is the Internet, which is a large network of networks facilitating
communication between nodes all over the world. Internet is linked together by
nodes using a set of protocols called the Internet Protocol Suite
\cite{rfc-1122}. The functionality of the protocol suite is organized into four
abstraction layers outlined in the following paragraphs.

\subsection{Network Layers}

The Internet Protocol Suite is organized into four layers, each one built upon
the one below it as shown in \cref{figure-network-layers}.

%%%%%%%%%%%% Internet Protocol Suite %%%%%%%%%%%%%%%%%%%
\begin{table}[h]
\begin{tabularx}{\textwidth}{| X |}
\hline
  \textbf{Application Layer} \\ \hline
  \textbf{Transport Layer} \\ \hline
  \textbf{Internet Layer} \\ \hline
  \textbf{Link layer} \\ \hline
\end{tabularx}
\caption{The layers of the Internet Protocol Suite}
\label{figure-network-layers}
\end{table}

\subsubsection{Link Layer}

The lowest layer of the protocol suite is the link layer, where the link is the
physical component used to interconnect two adjacent nodes in a network.
Ethernet is an example of a link layer protocol facilitating the transfer of
data between two physically connected nodes.

\subsubsection{Internet Layer}

 Where the link layer is only concerned of moving data over a wire to an
 adjacent node, the Internet layer is concerned of how to deliver data all the
 way from a source to a destination, possible passing through multiple nodes on
 its way. It does not guarantee delivery of data, since data can be lost on the
 way to the destination, but provide a best-effort. Guaranteed delivery is
 usually handled by the higher network layers of the Internet Protocol Suite.
 The \gls{ip} is the protocol that enables the transfer of messages between two
 nodes in a network. Messages between two nodes are sent as IP packets and are
 routed through possibly multiple other nodes before it reaches its destination.
 This routing function is fundamental for the Internet, as it allows nodes
 to communicate without knowing the exact network path to each other.

To provide a
common transport mechanism for all types of transmissions links, NATO has
decided that data communication in NATO systems should be based on IP packets
\cite{nnec-study}.


 \subsubsection{Transport Layer}

In the Internet protocol suite model, the transport layer provides end-to-end
communication services to applications. It builds on top of the network layer,
and takes responsibility for sending data all the way from a process on a source
computer to a process on the destination computer. The by far most used
transport protocol is the \gls{tcp}, which provides reliable transport of data
to applications. With reliable transport we mean that if data in a transmission
is lost or received in the wrong order, this is handled by the transport
protocol. This provides an important abstraction for applications so that they
do not need to deal with these issues themselves.

\subsubsection{Application Layer}

The top layer of the Internet Protocol Suite is the application layer. Its role
is to serve communication services to applications. When we talk about
application layer protocols, we usually talk about protocols that applications
use to communicate with other applications. Application layer protocols use the
communication services the transport layer provides. Examples of application
layer protocols are \gls{http} and the \gls{ftp}, which both rely on \gls{tcp}
as the underlying transport protocol.

\subsection{Messaging Patterns}

A message pattern describes how applications communicate with each other. This
communication is referred to as \textit{messages}. There exist multiple
messaging patterns and in this chapter we look into protocols using two very
common approaches:

\subsubsection{Request-Response}

Request-response is a message pattern where a requester sends a request to a
system. The system then process the request and responds with a response
message.

\subsubsection{Publish-Subscribe}

Publish-subscribe is a message pattern where subscribers express their interest
in a type of messages, often called topics or classes. Message publishers create
messages of a certain classes and publish them without needing to know who are
actually subscribing to these types of messages.  Many publish-subscribe
system employ a \textit{message broker} as seen in
\cref{figure-message-brokers}. The message broker handles published messages
from publishers and receives subscriptions from subscribers. The broker can
perform various tasks, such as message filtering and prioritize queuing.

\begin{figure}[h]
\centering
\includegraphics[scale=0.6]{images/pubsub.pdf}
\caption{Message Brokers}
\label{figure-message-brokers}
\end{figure}


\subsection{Network Metrics}

When transferring data over a network the transfer is subject to many factors
that may affect the transmission. A message sent over the Internet pass through
communication infrastructure and equipment of different quality and properties.
Network metrics are used to describe various aspects of the data transfer from a
node to another. In this thesis we use the following metrics:

\begin{description}

\item[Data Rate] The data rate describes the speed that data can be transferred
between two nodes.

\item[Latency] Latency is the time from a node sends a packet to the destination
node receives it.

\item[Packet Errors] The number of packets in a transmission that have been
altered due to noise or interference.

\end{description}


\section{Web Services}
\label{web-services}

Web services are client and server applications that communicate over a network.
They can be used to realize the \gls{soa} principles, and are in widespread use
in both civilian and military systems. It is worth noting that the term \textit{Web
services} is a broad term and can be used to describe different types of
services available over a network. The most common usage of the term refers to
the \gls{w3c} definition of SOAP-based Web services, but could also refer to
more simple HTTP-based \gls{rest} services.

In this thesis we investigate optimization techniques that should support both
\gls{w3c} Web services and \gls{rest}ful web services.

\subsection{W3C Web Services}
\label{w3c-web-services}

\gls{w3c} has defined Web services as \cite{wrc-web-service}:

\paragraph{}
\textit{
    A Web service is a software system designed to support interoperable
    machine-to-machine interaction over a network. It has an interface described in
    a machine-processable format (specifically WSDL). Other systems interact with
    the Web service in a manner prescribed by its description using SOAP-messages,
    typically conveyed using HTTP with an XML serialization in conjunction with
    other Web-related standards.
}

\paragraph{}

This definition points out a set of standards that enable machine-to-machine
interactions. Web service interfaces is described in documents called WSDL, and
communication is based on sending XML-based SOAP messages. There exists many
definitions of Web services where the core principles are the same, but the
finer details may vary. \Cref{figure-w3c-web-services} illustrates the
fundamental principles. The Web service technology is a realization of the
\gls{soa} principles, and provides loose coupling and eases integration
between systems.

\begin{figure}[h]
\centering
\includegraphics[scale=0.6]{images/web_services.pdf}
\caption{W3C Web services}
\label{figure-w3c-web-services}
\end{figure}

These standards that together make W3C Web services are presented in the
following sections.

\subsubsection{\glsentrylong{xml}}

The \gls{xml}\cite{W3C-XML} is considered as the base standard for Web services.
An XML document consists of data surrounded by tags and is designed to be both
machine and user readable. Tags describe the data they enclose. The tags can be
standardized, which allows exchange and understanding of data in a standardized,
machine-readable way.


\subsubsection{\glsentrylong{wsdl}}

\gls{wsdl} is an XML-based interface definition language that describes
functionality offered by a Web service \cite{w3c-wsdl}. The interface describes
available functions, data types for message requests and responses, binding
information about the transport protocol, as well as address information for
locating the service. This enables a formal, machine-readable description of Web
service which clients can invoke.

\subsubsection{SOAP}

SOAP is an application level, XML-based protocol specification for information
exchange in the implementation of Web services \cite{w3c-soap}. Data
communication in SOAP is done by nodes sending each other SOAP messages. A SOAP
message can be considered as an "envelope" consisting of an optional message
header and a required message body. The header can contain information not
directly related to the message such as routing information for the message and
security information. The body contains the data being sent, referred to as the
payload.

SOAP is transport protocol agnostic, which means it can be carried over various
underlying protocols. The far most used transport protocol is HTTP over TCP, but
other protocols such as UDP and SMTP can be used as well.

\subsection{\glsentrylong{rest}}
\label{rest}

In the previous sections we looked into the standards and specifications that
compose W3C Web services. However, there also exist other types of Web services
which do not follow these standards. In 2000, the computer scientist Roy
Fielding introduced \gls{rest} where he presented a model of how he thought the
Web \textit{should} work. This idealized model of interactions within a Web
application is what we refer to as the REST architectural style
\cite{rest-fielding}. REST attempts to minimize latency and network
communication while maximizing the independence and scalability of component
implementations. This is done by placing constraints on connector semantics
rather than on component semantics like W3C Web services.  REST is based on a
client-server model where a client requests data from a server when needed.

Web services that adhere to the REST style are called RESTful Web services. They
are closely associated with HTTP and use HTTP verbs (e.g. GET, POST, DELETE) to
operate on information located on a server. RESTful Web services typically
expose some sort of information, called resources in REST. \Cref{table-rest}
illustrates how a component exposes a set of operations of an example car
resource. Resources are identified by a resource identifier. While W3C Web
services are service oriented, we can look at REST as being more resource
oriented.

 %%%%%%%%%%%% REST eksempel %%%%%%%%%%%%%%%%%%%
 \begin{table}[h]
 \begin{tabularx}{\textwidth}{| X | X | X |}
 \hline
   \textbf{Resource identifier} & \textbf{HTTP Method}  & \textbf{Meaning}\\ \hline
   /vehicles/cars/1234 & GET & Return a car with ID 1234 from the system. \\ \hline
   /vehicles/cars/ & POST & Create a new car which will be added to the list of cars. \\ \hline
   /vehicles/cars/1234 & DELETE & Delete a car with ID 1234 from the system. \\ \hline
 \end{tabularx}
 \caption{Example of REST operations}
 \label{table-rest}
 \end{table}

 \gls{rest} is easy to understand and has gained a lot of traction in the civil
 industry in the latest years. Although NATO has chosen W3C Web services as the
 technology to do information exchange, REST is identified as a technology of
 interest to certain groups in NATO \cite{johnsen-recommendations}. One potential
 downside to NATO with REST however, is that RESTful Web services lack
 standardization, which may cause interoperability issues.

Closely associated with REST and the most used transport protocol for W3C Web
services are \gls{http}, which we present in detail in the next section.


\section{\glsentrylong{http}}

As we have seen in the previous sections, both RESTful and W3C Web services rely
on \gls{http} as the means of communicating with other services. The usage of
\gls{http} is very widespread and it is the foundation of data communication for
the \glsentrylong{www} since the early 90's. Its protocol specification is
coordinated by \gls{ietf} and the \gls{w3c}, and is defined as \cite{rfc-2616}:

\paragraph{}
\textit{
    The Hypertext Transfer Protocol (HTTP) is an application-level
    protocol for distributed, collaborative, hypermedia information
    systems. It is a generic, stateless, protocol which can be used for
    many tasks beyond its use for hypertext, such as name servers and
    distributed object management systems, through extension of its
    request methods, error codes and headers
}

\paragraph{}

\gls{http} started out as a simple protocol for raw data transfer across the
Internet and has since been updated in HTTP/1.0, HTTP/1.1 and most recently a
major update in HTTP/2.0. It is a request-response protocol which means that all
data exchanges are initiated with a client invoking a HTTP request and then
waits until a server responds with a HTTP response. A HTTP request consists of
the request method, \gls{uri}, protocol version, client information, and a optional
body. The server responds with a message containing a status line, protocol
version, a code indicating the success or error of the request, and a optional
body. Both HTTP requests and responses use a generic message format and can
contain zero or more HTTP headers. Headers are used to provide information about
the request/response or about the message body, e.g. information about the encoding
and caching information.

HTTP, being an application level protocol, relies on a transport protocol to
actually transfer data to another machine. HTTP communication most often, but
not necessarily, occurs over TCP/IP connections. The only requirement in the HTTP
specification is that a reliable transport protocol is used.

\subsection{HTTP Methods}

 Associated with all HTTP requests is a request method, which indicates the
 desired action to be performed on a resource located on a Web server. The set
 of HTTP methods defined in HTTP/1.1 is listed in \cref{table-http-methods}.

 %%%%%%%%%%%% HTTP methods %%%%%%%%%%%%%%%%%%%
 \begin{table}[h]
 \begin{tabularx}{\textwidth}{| X | X |}
 \hline
   \textbf{HTTP Method} & \textbf{Purpose} \\ \hline
   OPTIONS & Asks the server which HTTP methods and header fields it supports. \\ \hline
   GET & Retrieve information identified by the resource identifier (Request-URI). \\ \hline
   HEAD & Identical to GET, except that the HTTP body is not returned from the server. \\ \hline
   POST & Asks the server to accept the message payload from the client as a new resource.\\ \hline
   PUT & Similar to POST but allows the client to ask the server to update a resource identified by the request URI. \\ \hline
   DELETE & Requests that the resource identified by the request URI is deleted. \\ \hline
   TRACE & Echoes the HTTP request. Used for debugging. \\ \hline
   CONNECT & For use with a proxy that can dynamically switch to being a tunnel.\\ \hline
 \end{tabularx}
 \caption{HTTP methods}
 \label{table-http-methods}
 \end{table}

\section{\glsentrylong{tcp}}
\label{tcp}

\gls{tcp} is called the workhorse of the Internet because it is so critical for
how the Internet works. It is the primary transport protocol of the Internet
Protocol Suite\cite{rfc-1122} and provides reliable, in-sequence delivery of
two-way traffic (full-duplex) data. \gls{tcp} was defined in RFC
793\cite{rfc-793} back in September 1981 and has since been improved in various
RFC's. The main motivation behind \gls{tcp} was to provide reliable end-to-end
byte streams over unreliable networks. HTTP and other application layer
protocols often use TCP as their transport layer protocol. In the coming
sections we therefore present TCP in detail and discuss some of the issues we
may encounter using it.

 \subsection{The Protocol}

 TCP is a connection-oriented protocol, which means that a connection between a
 sender and the receiver must be established before any data can be transferred.
 TCP does this by using a three-way handshake to establish a connection. For
 each connection TCP initializes and maintains some status information such as
 window size, socket information and sequence numbers.

Computers supporting TCP have a piece of software which manages TCP streams and
interfaces to the Internet layer. Most often this software is a part of the
kernel \cite{computer-networks}. It accepts data streams from local processes,
and breaks them up into pieces, before sending them to the Internet layer. The
pieces are called TCP segments, which consist of a fixed 20 byte header,
followed by zero or more data bytes. The TCP software decides how big the
segments should be, but for performance reasons they should not exceed the
\gls{mtu} of the link (the physical network). Each segment should be so small
that it can be sent in a single, unfragmented package over the entire network.
This usually limits the size of each segment to the \gls{mtu} of the Ethernet,
which is 1500 bytes.

When the TCP software receives data from applications, it is not necessarily
sent immediately as it may be buffered before its sent. At the receiving end,
data is delivered to the TCP software, which reconstructs the original byte
stream and delivers it to the destination application.


\subsection{TCP Reliability}

When transferring data over the Internet, the data may pass through various
networks, routers and physical networks. Some of the routers may not work
correctly, a bit may be flipped when transferring data wirelessly, or some other
factor may come in to play. For these reasons, we have to accept that some of
the data will be damaged, lost, duplicated or delivered out of order.

TCP recovers from such faults by assigning sequence numbers to each packet being
sent. It then requires a positive acknowledgement from the receiver that the
data was actually received. If the acknowledgement is not received within a
timeout interval, the data is transmitted again. For the receiver the sequence
numbers are used to ensure that data is received in the correct order, as well
as eliminating duplicates. Furthermore, to detect damaged data, TCP applies
checksums to each segment transmitted. At the receiver the checksum is then
checked and damaged segments are discarded.

\subsection{Flow Control}

If a fast receiver sends data faster than a slow receiver is able to process,
the receiver will be swamped with data and may experience serious performance
reduction. Flow control is a mechanism to manage the rate of the data
transmission to avoid overflowing a receiver. TCP provides this by using a
window of acceptable sequence numbers that the receiver is willing to accept.
With every acknowledgement sent back to the sender, the window is specified.
This allows the receiver to control which segments, and how fast, the sender
can send.

\subsection{Congestion Control}

Congestion control is about controlling the data traffic entry into a network in
order to avoid network congestion. On its way from the sender to the receiver,
an IP packet may pass through different subnets with different capabilities.
Network congestion may occur if a node in a network receives more data than it
is able to pass forward. The consequence of this is that an increase in network
traffic to this node, would only lead to a small increase, or even a decrease,
of the network throughput \cite{Al-Bahadili2012}.

To avoid congestion, TCP uses a number of mechanisms. These aim to control the
rate of data packets entering into the networks to avoid congestion, but still
get as high throughput as possible. One of these mechanisms is
\textit{slow-start}, which general idea is to start transmitting with a low
packet rate, then gradually increasing the packet rate. When TCP notice that a
packet is eventually lost, it considers it as a sign of network congestion and
reduces the rate it send packets.

\subsection{Issues Using TCP in DIL}
\label{section:tcp-problems}

DIL networks are characterized by their high delay, low data rate and relatively
high error rate. Since TCP's congestion control interprets this as evidence of
congestion, it will back off and send with a lower packet rate. This could cause
TCP to send with a lower rate than the network actually can provide. Moreover,
it could also ultimately lead to the TCP connection terminating due to those
effects \cite{nato-disadvantaged-grids}.


\section{Protocols of Interest}

Since \gls{tcp} may be sub-optimal or even break down entirely in DIL networks,
we are in this thesis looking into alternative protocols and other optimization
techniques. In networks with low data rate, protocols with low overhead per IP
packet are beneficial. With frequent disconnects, protocols that are
connection-less may be more suitable than connection-oriented. One important
limitation is that NATO has chosen the "everything over IP", which means that
all optimization must occur on the top of the network layer. Because of this we
evaluate protocols in the transport and application layer.

In the following sections we give a short introduction to the protocols we are
investigating in this thesis. The protocols have been selected because of their
prevalence in the civil and military world or their reported performance in the
\gls{iot}. We get started by discussing the \gls{udp}, which alongside \gls{tcp}
is one of the core protocols of the Internet protocol suite.


\subsection{\glsentrylong{udp}}

The Internet has two main protocols in the transport layer, namely \gls{udp} and
\gls{tcp}. They have fundamentally different characteristics and use cases,
which we go through in this section. \gls{udp} was formally defined in 1980 in
RFC 768\cite{rfc-udp} and is a simpler protocol than \gls{tcp}. It sends
messages, called datagrams, to nodes over the \gls{ip} network. While \gls{tcp}
provides reliable transmission along with flow control and congestion control,
does UDP only support the sending of IP datagrams. Furthermore it is a
connectionless protocol, which means that the protocol can send messages
\textit{without} establishing a connection first. Since \gls{udp} does not
provide guaranteed delivery or in-order delivery of messages, it should only be
used by applications that do not require this.

To summarize, \gls{udp} is a more lightweight protocol than TCP. It has smaller
headers and less overhead. The downside is that it does not provide any
mechanisms for congestion control or reliability. UDPs lack of end-to-end
congestion control may result in drastic unfairness if an \gls{udp} stream is
competing with a \gls{tcp} stream \cite{floyd-congestion}. While a TCP stream
will detect congestion and back-down its traffic, an UDP stream will greedily
send at full-throttle, thus causing an unfair share of the available network.
UDP is therefore often referred to as not \textit{TCP-Friendly}.

 It is worth noting that UDPs lack of reliability may by handled on a higher
 level in the application stack on top of \gls{udp}. This is done by the next
 protocol we are looking into.

\subsection{\glsentrylong{coap}}

\gls{coap} is a specialized Web transfer protocol designed for use with
constrained nodes and  networks \cite{rfc-7252}. It is intended for
machine-to-machine applications typically found in the Internet of Things.
Furthermore, it is designed with a similar interface as HTTP in order to easily
integrate with Web services. \gls{coap} and HTTP work similar in the way that
they both use a client-server interaction model. \Gls{coap} is based on the REST
style where a server makes resources available under a \gls{uri}. Clients can
then interact with these resources using a subset of the HTTP-verbs: GET, PUT,
POST and DELETE.

CoAP messaging is based on asynchronously exchanging CoAP messages over UDP
between two endpoints. The current specification defines four types of CoAP
messages where each message uses a 4 byte fixed-length binary header.
\Cref{table:coap-messages} lists the four types of CoAP messages. A CoAP header
may be followed by \textit{options} and a payload. \Gls{coap} provide mechanisms
for optional reliability since \gls{udp} itself does not guarantee reliable
delivery. This is done by sending messages marked as \textit{Confirmable}, and
retransmitting using a default timeout and exponential back-off until an
\textit{Acknowledgement message} is eventually received. Basic congestion
control is done by strictly limiting the number of allowed outstanding requests
between a client and a server. The back-off mechanism also provides basic
congestion control.

\begin{table}[h]
\centering
\begin{tabularx}{\textwidth}{|X|X|}
\hline
\textbf{CoAP message}   & \textbf{Purpose}                                                                                                  \\ \hline
Confirmable Message     & CoAP message that  requires an acknowledgement. Used to provide reliable transport.                               \\ \hline
Non-confirmable Message & Used when no acknowledgement is wanted.                                                                           \\ \hline
Acknowledgement Message & Acknowledges that a specific Confirmable Message has arrived.                                                     \\ \hline
Reset Message           & Indicates that a Confirmable Message or a Non-confirmable Message was received, but not understood by the client. \\ \hline
\end{tabularx}

\caption{CoAP messages}
\label{table:coap-messages}
\end{table}

Typical CoAP messages may be small payloads from Internet of Things devices such
as temperature sensors, light switches etc. The CoAP specification states that a
CoAP message \textit{should} fit within a single IP packet to avoid IP
fragmentation. However, occasionally larger messages are needed. Therefore a
blockwise transfer technique has been proposed as an extension to CoAP in an
Internet Draft \cite{draft-coap-blockwise}. The block option allows for sending
larger messages in a block-wise fashion.

\subsection{\glsentrylong{amqp}}

The \gls{amqp} is an application layer protocol for sending messages.
It supports both the request-response and the publish-subscribe communication
paradigms. \gls{amqp} uses \gls{tcp} as its underlying reliable transport layer
protocol.

An important observation about AMQP is that it has two major versions which are
fundamentally different, version 0.9.1 and 1.0. The latter has been standardized
by \gls{oasis}, and is a  more narrow protocol as it only defines the network
wire-level protocol for the exchange of messages between two endpoints
\cite{oasis-amqp}. The concept wire-level protocol refers to the description of the format
of data sent over a network in form of bytes. Another difference between the
versions is that version 1.0 does not specify the details of broker
implementation. We investigate version 1.0 since it is the newest and has been
standardized.

An AMQP network consists of nodes connected via \textit{links}. Nodes can be
producers, consumers and queues. Producers generate messages, consumers process
messages, while queues store and forward them. These nodes live inside
\textit{containers}, which can be client applications and brokers. Each
container can have multiple nodes. AMQP version 1.0 does not specify the
internal workings of those nodes, but defines the protocol for transferring
messages between them. The basic data unit in AMQP is called a \textit{frame}
and is used to initiate, control and tear down the transfer of a message between
two nodes. The 9 different frames are listed in \cref{table-amqp-frames}.

AMQP is a connection-oriented since an AMQP connection must be established prior
to any communication. A connection is divided into independent unidirectional
\textit{channels}. An AMQP \textit{session} correlates two unidirectional
channels to form a bidirectional, sequential conversation between two
containers. To establish a connection the first operation is to establish a TCP
connection between the nodes. Then the protocol header is exchanged, allowing
the nodes to agree on a common protocol version. This is exchanged in plaintext
(not in an AMQP frame). The message itself is sent with the \textit{transfer}
frame. Larger messages can be split into multiple frames.


\begin{table}[h]
\begin{tabularx}{\textwidth}{| X | X |}
\hline
  \textbf{AMQP Frame} & \textbf{Purpose} \\ \hline
  Open & Describes the capabilities and limits of the node \\ \hline
  Begin & Begin a session on a channel \\ \hline
  Attach & Attach a link to a session \\ \hline
  Flow & Update link state \\ \hline
  Transfer & Transfer a message \\ \hline
  Disposition & Inform remote peer of delivery state changes \\ \hline
  Detach & Detach the link endpoint from the session \\ \hline
  End & End the session\\ \hline
  Close & Signal a connection close\\ \hline
\end{tabularx}
\caption{AMQP Frames}
\label{table-amqp-frames}
\end{table}

\subsection{MQTT}

MQTT is a publish-subscribe messaging transport protocol \cite{oasis-mqtt}. It
emerged in 1999 and recently became an \gls{oasis} standard in 2014. MQTT is
considered to be light weight and simple to implement, making it suitable for
use in networks where the data rate is limited and/or a low code footprint is
needed. These properties make MQTT popular as a \gls{iot} protocol. The protocol
is broker-based and runs on top of the TCP/IP protocols.

MQTT provides message sending services to applications and offers different
levels of \gls{qos}, specifying the delivery policies for a message. This is
beneficial in networks where messages may be lost while traveling through a
network. The lowest level of \gls{qos} is \textit{at most once}, which specifies
that a message should arrive at the receiver either once or not at all. Next,
the policy \textit{at least once} ensures that the message arrives at the
receiver at least once, but possible multiple times. The last and highest level
of MQTT's \gls{qos}, \textit{exactly once}, guarantees one, and only one,
delivery of the message. The protocol works by sending different MQTT control
packets, listed in \cref{table:mqtt-packets}. Only \textit{exactly once}
\gls{qos} requires the usage of the control packets PUBREC, PUBREL and PUBCOMP.

\begin{table}[h]
\begin{tabularx}{\textwidth}{| X | X |}
\hline
  \textbf{MQTT Control Packet} & \textbf{Purpose} \\ \hline
  CONNECT & Client requests a connection to the Server \\ \hline
  CONNACK & Acknowledge connection request \\ \hline
  PUBLISH & Publish a message \\ \hline
  PUBACK & Publish acknowledgement \\ \hline
  PUBREC &  Publish received \\ \hline
  PUBREL & Response to a PUBREC Packet \\ \hline
  PUBCOMP & Publish complete \\ \hline
  SUBSCRIBE & Subscribe to topics \\ \hline
  SUBACK & Subscribe acknowledgement\\ \hline
  UNSUBSCRIBE & Unsubscribe from topics\\ \hline
  UNSUBACK & Unsubscribe acknowledgement \\ \hline
  PINGREQ & PING request \\ \hline
  PINGRESP & PING response \\ \hline
  DISCONNECT & Disconnect notification \\ \hline
\end{tabularx}
\caption{MQTT Control packets}
\label{table:mqtt-packets}
\end{table}


\subsection{\glsentrylong{sctp}}

The \gls{sctp} is a transport-layer protocol, which offers functionality from
both \gls{udp} and \gls{tcp} \cite{rfc-sctp}. The motivation behind the protocol
is that many developers find TCP too limiting, but still require more
reliability than UDP can provide. \gls{sctp} tries to solve these issues. It is
message-oriented like UDP, but ensures reliable, in sequence transport of
messages with congestion control like TCP. \gls{sctp} is a connection-oriented
protocol and provide features like multi-homing and multi-streaming.
Multi-homing is the possibility to use more than one network path between two
nodes. This increases reliability since if one path fails, messages can still be
sent over the other links. Multi-streaming refers to \gls{sctp} ability to
transmit several independent streams of data at the same time, for example
sending an image at the same time as a HTML Web page.

\begin{figure}[h]
\includegraphics[scale=0.5]{images/sctp.pdf}
\caption{Overview of SCTP}
\end{figure}



\section{Summary}

In this chapter we presented computer networks in general, before we discussed
the two most common types of Web services. Moreover, we discussed the protocols
that these Web services use in order to transmit messages over the Internet. We
also introduced some new protocols designed to work in "Internet of things"
networks, which have many of the same characteristics as DIL networks. The
protocols are summarized in \cref{table:protocols:summary}.

%%%%%%%%%%%% Transport protokoller %%%%%%%%%%%%%%%%%%%
\begin{table}[h]
\begin{tabularx}{\textwidth}{| l | l | X |}
\hline
  \textbf{Protocol} & \textbf{Network layer} & \textbf{Summary} \\ \hline
  TCP & Transport. & Stream-oriented transport protocol. Reliable and with congestion control. \\ \hline
  UDP & Transport. & Message oriented. Low overhead, but lacks reliability and TCP-friendliness. \\ \hline
  SCTP & Transport. & Similar to UDP but also provide reliable, in sequence transport of messages like TCP. \\ \hline
  HTTP & Application. Uses TCP. &  Widely used and the foundation for World Wide Web. \\ \hline
  CoAP & Application. Uses UDP. & Low header overhead with optional reliability. \\ \hline
  AMQP & Application. Uses TCP. &  Messaging middleware with store-and-forward capabilities.\\ \hline
  MQTT & Application. Uses TCP & Light weight and simple pub/sub protocol. \\ \hline
\end{tabularx}
\caption{Summary of protocols}
\label{table:protocols:summary}
\end{table}

Many of the mentioned protocols have been previously researched for use in DIL
networks. In the next chapter we will present relevant work in this area.
