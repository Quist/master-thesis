\chapter{Related Work}

In this chapter we will discuss earlier relevant work in the area of improving
the performance of Web services in \gls{dil} environments. We identify results
and recommendations that are applicable to this thesis. Furthermore, we discuss
existing proxies and what they offer.

Quite an amount of research has been done in the area of compression. Also,
improving the performance of Web services in \gls{dil} environments has been
explored, but mostly for SOAP-based Web services.

In the report IST-090, a task group investigated solutions for making \gls{soa}
applicable at the tactical level. Three key issues that needs to be addressed in
order to apply Web services in tactical networks were
identified\cite{ist-118,ist-090}:

\label{section:DIL-problems}

\subsubsection{End-to-end connections}

Web services depend on a direct, end-to-end connection between the client and
the service. Attempting to establish and maintaining connections in DIL
environments can lead to increased communication overhead and possible complete
breakdown of communication. Most Web services use TCP as the transport protocol,
which is a connection-oriented protocol designed for wired networks. In DIL
environments with high error rates and high latencies, the congestion control of
TCP will cause sub-optimal utilization of the network due to frequent connection
timeouts. Similar, HTTP, which is the application layer protocol most often
used together with TCP, struggles in such environments. HTTP is a synchronous
protocol, which means that the HTTP connection is kept open until a response is
received. Long response times cause timeouts. IST-090 points out the obvious
solution to replace HTTP and TCP with other, more suitable protocols.

The report\cite{ist-090} mentions two approaches to replace HTTP/TCP. The
clients and services themselves can be modified to support other protocols, or
proxies which support alternative protocols can be used. With employing a proxy
solution, standards compliance can be retained.


\subsubsection{Network heterogeneity}

Another issue is when heterogeneous networks are interconnected. Different
performance in networks may lead to buildup of data in buffers, risking loss of
information. A proposed solution to this is to have store-and-forward support,
which can support that messages are not dropped, but stored and forwarded when
possible.


\subsubsection{Web service overhead}

Optimization approaches should seek to reduce the network traffic generated by
Web services by using techniques as compression to reduce the size of messages.
Another approach is to reduce the number of messages being sent, which was
looked into in IST-090\cite{ist-090}. In their work they investigated three
different ways to do this:

\begin{enumerate}
    \item Employing caching near the client in order to reuse older messages.
    \item Using publish/subscribe paradigm, which allows clients to subscribe to
    information instead of requesting it. This allows the same message to be sent
    to multiple clients.
    \item Employing content filtering, which filters out unnecessary data.
\end{enumerate}


\section{Proxies}

This section lists previous implementations of proxies designed to work in
\gls{dil} environments.


\subsection{DSProxy}

DSProxy is a proxy solution developed by \gls{ffi}, which transports SOAP
messages over DIL networks. It reduces bandwidth needs by employing different
optimization techniques such as compression. It also provides delay tolerance,
which allows \gls{cots} clients to function in DIL networks.


\subsection{AFRO}

\gls{afro} is an edge proxy which offers different levels of \gls{qos} to Web
Services through performance monitoring and application of the context-aware
service provision paradigm. It perform so called adaption actions, which
modifies the SOAP XML messages by changing their encoding to more efficient data
representation. It also cuts out information that is accepted to be removed by
the service requester. However, since the proxy modifies the data being sent,
the checksum of the data is also changed. In applications where we want to be
sure that no one has tampered with the data before arriving, checksums are often
used. Therefore this solution would not work for such applications.


\subsection{Suri}
To be done.

\section{Evaluation of Transport Protocols}

Previous studies have investigated potential gains from replacing HTTP/TCP
with alternative transport protocols
\cite{evaluation-transport-protocols-web-services}. They looked into how TCP,
UDP, SCTP and AMQP for conveying Web services traffic under typical military
networking conditions.

\begin{itemize}
    \item \gls{sctp} has the highest success rate in military tactical
    communication. However, on the lower bandwidth links the protocols tends to
     generate more overhead than TCP. Due to SCTP has a more complex connection
     handshake procedure and in addition use heartbeat packets.
\end{itemize}

\subsection{Configuration}
IST-90: Configure HTTP on the application server or ESP to prevent time-outs.
Anbefaler at hvis man trenger å gjøre propetiære optimaliseringer, så burde de
plasseres i proxier.

% Heller kanskje referere til ist-118


\section{Compression}

Data compression is the technique of encoding information using fewer bits than
the original representation. The goal is to reduce data transmission time or the
storage requirements. We divide compression lossy and lossless compression.
Lossy compression is used to compress data such as images and movies where the
consequence of loosing some of the data is not critical. Lossless compression
utilize repeating patterns in the  data in order to represents the same data in
a more efficient way.

\gls{xml} is the data format used by Web services and has a significant
overhead. Previous studies have evaluated different compressions algorithms for
Web services. Previous experiments shows EFX has the best compression results with
GZIP as the second best alternative\cite{johnsen-trude-compression-techniqes}.


\section{Summary}

As discussed in this chapter, there exist research and experiments targeting
SOAP-based Web services in \gls{dil} environments. Proxies have been created but
their are either limited to SOAP-based Web services or are inadequate to be used
due to security reasons.

%%%%%%%%%%%% Proxy approaches %%%%%%%%%%%%%%%%%%%
\begin{table}
    \begin{tabular}{|l|l|}
    \hline
    \textbf{Approach}          & \textbf{Summary}     \\ \hline
    Build from scratch & Time consuming, complex.       \\ \hline
    Apache Camel      & Supports some of the protocols \\ \hline
    Squid      & \\ \hline
    \end{tabular}
    \caption {Possible proxy approches}
\end{table}
