\chapter{Testing and Evaluation}

In this chapter we present how the testing and evaluation of the proxy was
performed and present the results we obtained.  The goal is to measure any
possible improvements(or deterioration) of the performance of Web services when
the proxy developed as a part of this thesis is being used. Since the proxy was
developed as a prototype for military usage, we wanted to use test scenarios
that resembles actual military and civilian usage. For the purpose of testing,
we therefor originally developed two set of applications, one W3C Web service
and one RESTful Web service. These applications were then put to test in
networks with different characteristics. During testing, we discovered that some
of the protocols were very sensitive to the size of the messages being sent. We
therefor also developed a complementary test service which allowed us to test
sending messages of different sizes.

We'll get started by discussing the test and evaluation tools used, before we
introduce the different test applications, test cases and the different types of
networks used for testing. Then we present the test results for each of the
three aspects of DIL, \textit{disconnected, intermittent and disconnected.}



To evaluate how different network properties affects performance, the tests
was performed on networks with different characteristics. The base case was to
test without any intentional limitations to the network and without the actual
usage of the proxy. Then we introduced usage of the proxy and evaluated it in
different types of networks.
Furthermore we performed tests with two setups, first with machine-to-machine
over an Ethernet cable, then over actual military communication equipment. The
usage of actual military equipment allowed us to get as realistic results as
possible.

\section{Types of DIL networks}

Military communication can occur over a wide range of different technologies and
environments. These include satellite link networks(SATCOM), line-of-sight(LOS),
combat network radio(CNR) and WIFI. Wifi is divided into two types to illustrate
both with good connection and one with less. Some communication technology, such
as Satellite communication, is characterised by long communication delay while
others may be by their low data rate. An overview of military communication
technologies can be seen in \cref{figure-networks-overview}.

\begin{figure}[h]
\includegraphics[scale=0.25]{images/networks_overview.pdf}
\caption{Overview of tested networks}
\label{figure-networks-overview}
\end{figure}

An infinite number of possible network combinations exists, so we have in this
thesis chosen to focus on five different network types identified by the task
group IST-118 for DIL-testing. We also investigated \gls{lte}, commonly known as
4G, a network technology which has become in widespread use in the latest years.
The reason for including LTE in addition to the ones from IST-118, is that the
Norwegian Defense is looking into the possibility of using LTE. Thus making it
interesting for us to investigate the performance under this type of network as
well. However, we eventually found out that LTE has gotten so fast and reliable,
that it is not really relevant from a DIL perspective. We therefor instead
looked into \gls{edge}, which is used as a fall back in geographical areas where
\gls{lte} and 3G is not available. The different networks and their properties
are summarized in \cref{table-network-types}.

\begin{table}[h]
\begin{tabular}{| l | l | l | l | l |}
\hline
  \textbf{Network} & \textbf{Data Rate} & \textbf{Delay} & \textbf{PER} \\ \hline
  Satellite Communication & 250 kbps & 550 ms & 0 \% \\ \hline
  Line of Sight & 2 mbps & 5 ms & 0 \% \\ \hline
  Wireless Fidelity (WiFi) 1 & 2 mbps & 100 ms & 1 \% \\ \hline
  WiFi 2 & 2 mbps & 100 ms & 20 \% \\ \hline
  Combat Net Radio with Forward Error Correction & 9.6 kbps & 100 ms & 1 \% \\ \hline
  Edge & 125 kbps & 200 ms & 0 \% \\ \hline
\end{tabular}
\caption{Different network types}
\label{table-network-types}
\end{table}


\section{Testing and Evaluation Tools}

In order to evaluate how our solution impacts the performance of Web services in
DIL environments, we needed some way of simulating such environments. Obviously,
we would have got the most realistic test environment by testing "out in the
field" ourself. However, this would require of a considerable amount of effort
and it would be difficult to reproduce the exact same environment and test
results. We therefor choose to instead emulate DIL networks. For testing we used
two approaches, the first one connecting two machines through a third machine.
The third machine used a component in the Linux kernel to control the flow
of the network traffic flowing through it, allowing us to simulate DIL networks.
The second approach involved using actual military equipment in a laboratory at
FFI. The benefit of using actual equipment, is that we got as realistic tests as
possible.


\subsection{\glsentrylong{netem}}

The Linux kernel offers a rich set of tools for managing and manipulating the
transmission of packets. \textbf{tc}(traffic control) is a Linux program to
configure and control the linux kernels Network scheduler. \gls{netem} is an
enchancement of the traffic control facilities that allows us to control delay,
packet loss and other characteristics to packets outgoing from a selected
network interface\cite{man-netem}. These tools allow us to emulate many of the network
characteristics that makes DIL.

%Siter man-page om tc-netem
%Siter tldp -> Traffic Control HOWTO

\subsubsection{Delays}

NetEm can emulate delays on packets on a specific link. In
\cref{listing-netem-delay} we add a fixed delay on 100 ms to all packets going
out of local Ethernet.

\begin{lstlisting}[frame=single, caption="Emulating delay", label=listing-netem-delay]
  tc qdisc add dev eth0 root netem delay 100ms
\end{lstlisting}

\subsubsection{Corrupt rate}

The corrupt rate allows us to insert random data into a chosen percent of
packets.

\subsubsection{Data rate}

NetEm can set the data rate by delaying packets based on their packet size.

\subsection{Iperf 3}

iperf is a tool for performing network throughput measurements. Together with
ping we, used this tool to confirm that the \gls{netem} configuration worked as
expected.

\subsection{Wireshark}

Wireshark is a packet analyser and allows for network analysis and let us see
the network traffic. Using this tool, we could investigate the behaviour of each
protocol used for testing.



\section{Test Setup}
\label{testing-environment}

The majority of testing was performed at the FFI-lab at Kjeller. All the test
applications consisted of one client and one Web service, where the client would
request the service for some sort of data. The client were hosted on one
computer and the service at an another computer. The majority of testing was
done using NetEm to emulate DIL networks, and some testing was done using actual
military radios. The machines used for testing is listed in
\cref{table-machines}.

\begin{table}[h]
\begin{tabular}{| l | l | l |}
\hline
  \textbf{Role} & \textbf{OS} & \textbf{Kernel version}\\ \hline
  Client & Debian & x \\ \hline
  Web service & Ubuntu & x  \\ \hline
  Router & Ubuntu & x \\ \hline
\end{tabular}
\caption{Machines involved in the testing}
\label{table-machines}
\end{table}

\subsection{NetEm Setup}

In this setup, the client and Web service machines were connected to each other
through a third computer, acting as a router. This router machine had two
network cards and networked together the other machines by Ethernet cables. The
setup can been seen in \cref{figure-testing-environment}. In order for the
router machine to forward IP packets back and forth between the client and
server, IP forwarding was enabled on the kernel.

\begin{figure}[h]
\includegraphics[scale=0.6]{images/testing_environment.pdf}
\caption{Testing environment}
\label{figure-testing-environment}
\end{figure}

The server and client are assigned an IP address in two different subnets.
This is done by the Linux network interface administration program
\textit{ifconfig}. In \cref{listing-ifconfig-client} the client machine is is
assigned the IP address 192.168.2.44.

\begin{lstlisting}[frame=single, caption="Configuring a network interface of the router", label=listing-ifconfig-client]
ifconfig eth0 192.168.2.1 up
\end{lstlisting}

After setting up the IP addresses we need to configure the routing so that the
kernel know where to route the network traffic. In this case we want all
traffic to go through the routing machine. In \cref{listing-routing} we
configure all IP traffic boud for the subnet 192.168.1.X to be routed through
the router machine with IP 192.168.2.1.

\begin{lstlisting}[frame=single, caption="Configuring routing rules for the client", label=listing-routing]
ip route add unicast 192.168.1.0/24 via 192.168.2.1
\end{lstlisting}

\subsubsection{Emulating different types of networks}

Since all network traffic passes through the routing machine, we can control
the flow of IP packets here. As previously discussed, we use NetEm.  For each
network configuration, a bash script is run. This script configures the
network interfaces in order to get the correct network behaviour. Both
interfaces are configured so the network is symmetrical in both directions.

\subsection{Military Radio Setup}

Although testing on regular machines with emulated network gives us a good
indication, to get as realistic results as possible we also performed tests on
military communication equipment. The setup is illustrated in
\cref{figure-radio-testing-environment}.

\begin{figure}[h]
\centering
\includegraphics[scale=0.6]{images/radio_testing_environment.pdf}
\caption{Testing environment}
\label{figure-radio-testing-environment}
\end{figure}

\subsection{Proxy setup}

In order to enable the applications to tunnel all their HTTP traffic through our
proxy, we needed a way to setup a proxy without altering the applications
themselves. Fortunately, Java provide mechanisms to deal with
proxies\cite{oracle-proxy}. We configured the \gls{jvm} to get the applications
to tunnel all HTTP traffic through our proxy. This is done by setting properties
to the \gls{jvm}:


\begin{lstlisting}[frame=single, caption="Setting a proxy on the \gls{jvm}", label=test]
java -Dhttp.proxyHost=localhost \
-Dhttp.proxyPort=3001 \
-Dhttp.nonProxyHosts= \
-jar target/client.jar
\end{lstlisting}

In \cref{test} the application \textbf{client.jar} is started and all HTTP
traffic will go through the proxy server at localhost on port 3001.

\section{Test Execution}

For our tests we use originally used two different sets of applications. One for
W3C Web services and one for RESTful  Web services. While W3C web Services only
uses HTTP simply as a transport mechanism, REST utilizes the different HTTP
methods to indicate which operation to perform on a resource. Each test scenario
was therefor performed with both a W3C Web service application and RESTful Web
service application. Each service is deployed in Glassfish 4, while the client
is executued either from the command line or directly from the Netbeans IDEA.
Data being sent between the client and server is by default sent uncompressed.

During testing we discovered that especially CoAP was very sensentive to the
size of the messages being sent. We therefor developed a test application that
allowed the client to request a number of bytes from the server. This allowed us
to see how CoAP performed with different message sizes.

\subsection{NFFI W3C Web service}

For the purpose of testing W3C Web service applications we created a mock system
which allows a client to request a service to report positions of friendly
forces. The position reports use the \gls{nffi} format, which has an associated
XML schema with it. One test run is illustrated in \cref{figure-nffi-flow} and
consist of the client making a HTTP POST request to the Web service. Associated
with the request is an XML payload which tells the Web service which operation
to invoke. In our case, the service then returns an XML message containing a
large number of positions in the nffi format.

\begin{figure}[h]
\centering
\includegraphics[scale=0.6]{images/nffi_flow.pdf}
\caption{NFFI Web service}
\label{figure-nffi-flow}
\end{figure}


\subsection{RESTful car Web service}

The RESTful Web service is an example service keeping order of cars in a ``car
system''. The service exposes an \gls{api} which offers different operations to
manage the car system. Clients can invoke these operations by using HTTP
requests and utilizing the associated HTTP method to indicate what to do with an
resource. Since RESTful services are payload agnostic, we choose JSON to
represent the data being sent between the server and the client. JSON is a
lightweight data-format. Each test run consist of a client sequentially invoking
the server with different API requests. The most common HTTP-methods GET, PUT,
POST, and DELETE are all part of the testing. An example, not inclusive, test run
is illustrated in \cref{figure-rest-flow}.

\begin{figure}[h]
\centering
\includegraphics[scale=0.6]{images/rest_flow.pdf}
\caption{RESTful car service}
\label{figure-rest-flow}
\end{figure}


\subsection{Request size application}

This application allowed us to test with different message sizes.

\subsection{Test parameters}

The tests was performed with the following parameters.

\begin{itemize}
	\item GZIP compression on/off.
	\item Without and with proxies.
    \item Transport protocol used.
\end{itemize}


\section{Function tests}

The first phase of the testing was performed without any actual intended
limitations to the network. The objective of this testing is to validate that
the proxy is working correctly and have a benchmark to compare other results
with. This phase was again divided into two phases, one without the usage of
proxy and one with the usage of it. This allows us to investigate any potential
overhead associated with the usage of the proxy.

\subsection{Execution}

The Web service client and the service itself was started on separately machines
interconnected through a third machines acting as a router as discussed in
\cref{testing-environment}. However for the first phase, the client and server
did not use any proxy.

Warm-up.

Number of tests.

Delay less than 1 ms.
Iperf bandwidth: 7.76 Mbits/sec


\subsection{Results and Analysis}

Enabling compression yields an improvement in the performance. We also notice
that HTTP and CoAP has a almost identical performance, while AMQP has
significant longer RTT.

\begin{figure}[H]
\center
\includegraphics[scale=0.75]{../results/function_tests/nffi/out.pdf}
\caption{W3C Web services results}
\end{figure}

\begin{figure}[H]
\center
\includegraphics[scale=0.75]{../results/function_tests/rest/out.pdf}
\caption{REST results}
\end{figure}


\section{DIL Tests - Disconnected}

In this scenario we evaluate  the performance with the DIL characteristic
\textit{disconnected}, which refers to the network suddenly going down when the
application is sending data. The objective of this testing is to evaluate how
the proxy manages disconnects over longer periods of time. We define the success
criteria for this test to be that the client is able to eventually process his
request after the connection is reestablished. The client HTTP request should
not be interrupted in any way, other than it taking longer time to process the
request.

\subsection{Execution}

 The tests are performed on a unlimited network. During testing the Ethernet
 cable between the client machine and the router was removed for about 60
 seconds. It was then reconnected.

\subsection{Results}

For both the REST and W3C Web service test scenarios the results were identical.
Without using proxies, the connection timed out and the applications were unable
to continue. With proxies the connection did not time out, and the protocols
retransmission mechanism were able to continue transmission when connection was
reestablished.

\begin{table}[h!]
\begin{tabular}{| l | l |}
\hline
  \textbf{Test} & \textbf{Result} \\ \hline
  Without proxy & Connection timeout \\ \hline
  Proxy with HTTP & Success \\ \hline
  Proxy with AMQP & Success \\ \hline
  Proxy with CoAP & Success \\ \hline
\end{tabular}
\caption{W3C Web service results}
\end{table}

\begin{table}[h!]
\begin{tabular}{| l | l |}
\hline
  \textbf{Test} & \textbf{Result} \\ \hline
  Without proxy & Connection timeout \\ \hline
  Proxy with HTTP & Success \\ \hline
  Proxy with AMQP & Success \\ \hline
  Proxy with CoAP & Success \\ \hline
\end{tabular}
\caption{RESTful Web service results}
\end{table}



\section{DIL Tests - Intermittent}

\textit{Intermittent} refers to the network connection being lost, but then
regained again. The objective of this testing is to evaluate how the proxy
manages frequent temporary loss of connections. The success criteria is the same
as for disconnected, the client should not notice any disruption of service.

\subsection{Execution}

Not done yet. Similar to disconnect.

\subsection{Results}

\begin{table}[H]
\begin{tabular}{| l | l |}
\hline
  \textbf{Test} & \textbf{Result} \\ \hline
  Without proxy & Connection timeout \\ \hline
  Proxy with HTTP & Success \\ \hline
  Proxy with AMQP & Success \\ \hline
  Proxy with CoAP & Success \\ \hline
\end{tabular}
\caption{W3C Web service results}
\end{table}

\begin{table}[H]
\begin{tabular}{| l | l |}
\hline
  \textbf{Test} & \textbf{Result} \\ \hline
  Without proxy & Connection timeout \\ \hline
  Proxy with HTTP & Success \\ \hline
  Proxy with AMQP & Success \\ \hline
  Proxy with CoAP & Success \\ \hline
\end{tabular}
\caption{RESTful Web service results}
\end{table}

\section{DIL Tests - Limited}
The
different types of networks seeks to emulate properties of actual communication
devices used by the military.
The third DIL characteristic, \textit{limited}, refers to different ways a network
can be limited. This includes high delays, packet loss and low bandwidth.

The metrics we use to characterize a network is data rate, delay and \gls{per}.






In the following sections the test cases is ran for each network
configuration.


\subsection{Satellite communication}

In this test scenario we emulate \gls{satcom}. Low data rate, high delay.

\subsubsection{Execution}
Placeholder

\subsubsection{Results and analysis}
AMQP has a very long response time for both test scenarios while also CoAP
struggles with uncompressed xml-messages. Other

\begin{figure}[H]
\center
\includegraphics[scale=0.75]{../results/satellite/nffi/out.pdf}
\caption{W3C Web services results}
\end{figure}

\begin{figure}[H]
\center
\includegraphics[scale=0.75]{../results/satellite/rest/out.pdf}
\caption{REST results}
\end{figure}

\subsection{Line-of-Sight}

In this test scenario we emulate so-called \gls{los} networks, which are
characterized by being a radio-based type of network with no physical obstacles
between the nodes in the network. High data rate, low delay and zero error.

\subsubsection{Execution}
Placeholder

\subsubsection{Results and analysis}

\begin{figure}[H]
\center
\includegraphics[scale=0.75]{../results/los/nffi/out.pdf}
\caption{W3C Web services results}
\end{figure}

\begin{figure}[H]
\center
\includegraphics[scale=0.75]{../results/los/rest/out.pdf}
\caption{REST results}
\end{figure}

\subsection{WiFi 1}

About this type of network.

\subsubsection{Execution}
Placeholder

\subsubsection{Results and analysis}

\begin{figure}[H]
\center
\includegraphics[scale=0.75]{../results/wifi1/nffi/out.pdf}
\caption{W3C Web services results}
\end{figure}

\begin{figure}[H]
\center
\includegraphics[scale=0.75]{../results/wifi1/rest/out.pdf}
\caption{REST results}
\end{figure}


\subsection{WiFi 2}

About this type of network.

\subsubsection{Execution}
Placeholder

\subsubsection{Results and analysis}

\begin{figure}[H]
\center
\includegraphics[scale=0.75]{../results/wifi2/nffi/out.pdf}
\caption{W3C Web services results}
\end{figure}

\begin{figure}[H]
\center
\includegraphics[scale=0.75]{../results/wifi2/rest/out.pdf}
\caption{REST results}
\end{figure}

\subsection{Combat Net Radio with Forward Error Correction}

About this type of network.

\subsubsection{Execution}
Placeholder

\subsubsection{Results and analysis}

\begin{figure}[H]
\center
\includegraphics[scale=0.75]{../results/cnr/nffi/out.pdf}
\caption{W3C Web services results}
\end{figure}

\begin{figure}[H]
\center
\includegraphics[scale=0.75]{../results/cnr/rest/out.pdf}
\caption{REST results}
\end{figure}


\subsection{Edge}

About this type of network.

\subsubsection{Execution}
Placeholder

\subsubsection{Results and analysis}

\subsection{Kongsberg Radio}

About this type of network.

\subsubsection{Execution}
Placeholder

\subsubsection{Results and analysis}

\begin{figure}[H]
\center
\includegraphics[scale=0.75]{../results/kongsberg/nffi/out.pdf}
\caption{W3C Web services results}
\end{figure}

\begin{figure}[H]
\center
\includegraphics[scale=0.75]{../results/kongsberg/rest/out.pdf}
\caption{REST results}
\end{figure}




\section{Summary}

In this section the results from the tests are presented. These results lead up
to the discussion and conclusion in the next chapter.
