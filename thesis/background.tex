\chapter{Technical Background}

In this chapter we present concepts and protocols that are central for this
thesis.

We explore challenges a military network has. Next, we discuss common Web
services used for exchanging data in military systems. Then we look into a
number of protocols that we can replace HTTP/TCP with in order to increase the
performance of Web services. We also introduce some common optimization
techniques used to improve network performance. Finally, we present some
available frameworks and solutions for creating a proxy.

%%%%%%% WEB SERVICES
\section{Web services}
\label{web-services}
%TO-DO lede inn smooth mot soa.
Web services are client and server applications that communicate over a
network and can be used to implement a service-oriented architecture. Web
services are critical in any data systems and are in widespread use in both
civilian and military systems. It is a broad term and can be used to describe
different types of services that are available over a network. The most common
usage of the term refers to the \gls{w3c} definition of SOAP-based Web
services, but could also refer to more simple HTTP-based \gls{rest} services.

In this thesis we investigate optimization techniques that should support both
\gls{w3c} Web services and \gls{rest}ful web services.

\subsection{W3C Web services}

\gls{w3c} has defined Web Services as \cite{wrc-web-service}:

\paragraph{}
\textit{
    A Web service is a software system designed to support interoperable
    machine-to-machine interaction over a network. It has an interface described in
    a machine-processable format (specifically WSDL). Other systems interact with
    the Web service in a manner prescribed by its description using SOAP-messages,
    typically conveyed using HTTP with an XML serialization in conjunction with
    other Web-related standards.
}

\paragraph{}

This definition points out a set of standards that enables machine-to-machine
interactions. All communication is based on sending XML-based SOAP messages.
It exists many definitions of Web services where the core principles are the
same, but the finer details may vary.  These standards are discussed in the
following sections. The Web Service technology is realization of the \gls{soa}
principles which provides loose coupling and ease integration between systems.

Figur her.


\subsubsection{XML}

The \gls{xml} is considered as the base standard for Web services. An XML
document consist of data surrounded by tags and is designed to be both machine
and user readable. Tags describe the data they enclose. The tags can be
standardized, which allows exchange and understanding of data in a standardized,
machine-readable way.


\subsubsection{Service descriptions: WSDL}

\gls{wsdl} is an interface definition language that using XML describes
functionality offered by a Web Service. The interface describes available
functions, data types for message requests and responses and binding
information about the transport protocol, as well as address information for
locating the service. This enables a formal, machine-readable description of
Web Service which clients can invoke.


\subsubsection{SOAP}

\gls{soap} is an application level XML-based protocol specification for
information exchange in the implementation of Web services. It is transport
protocol agnostic and can be carried over various protocols. The far most used
transport protocol is HTTP over TCP, but other protocols such as UDP and SMTP
can be used as well. A SOAP message is an "envelope" consisting of an optional
header and a required body. The header can contain information not directly
related to the message such as routing information for the message and
security information. The body contains the data being sent, known as the
payload.

\subsubsection{W3C Web Service overhead}

W3C Web services are associated with a considerable amount of overhead. Web
Service technology is based on SOAP, which use XML-based messages. It is a
textual data format and produce much larger messages than binary formats.

% Referere  til IST-090 sluttarpport. IST-090 diskuterer alt.

\subsection{\glsentrylong{rest}}
\label{rest}

There also exist other types of Web services which does not follow the
previously discussed standards. \gls{rest} is an architectural style which let
users manipulate data using a set of stateless operations. It is based on a
client-server model where a client requests data from a server when needed. It
is closely associated with HTTP and use HTTP verbs(e.g GET, POST,
DELETE) to operate on resources on a server.

 \gls{rest} is easy to understand and has gained a lot of traction in the
civil  industry in the latest years. \gls{rest} uses exclusively HTTP over
TCP.  However, TCP does not necessarily perform satisfactorily in \gls{dil}
environments, which limits the usability in tactical networks(trenger kilde?).
It also lack standardization, which may cause interoperability issues.

%Frank tipser om FFI-rapport som kilde. Ref. samtale 14 jan 46:30


\section{HTTP}

\section{Transport protocols}


\section{Protocols of interest}

% Trengs en diskjon om NATO's everything over IP her. Man sier det i scope, men
%må diskutere det her.

In order to improve the performance of Web services in \gls{dil} environments we
have investigated the usage of alternative transport protocol other than TCP can
be utilized. In this thesis we're looking into protocols in the transport and
application layer of the Internet Protocol Suite\cite{rfc-1122}.

%%%%%%%%%%%% Internet Protocol Suite %%%%%%%%%%%%%%%%%%%
\begin{table}[h]
\begin{tabularx}{\textwidth}{| X |}
\hline
  \textbf{Application Layer} \\ \hline
  \textbf{Transport Layer} \\ \hline
  \textbf{Internet Layer} \\ \hline
  \textbf{Link layer} \\ \hline
\end{tabularx}
\caption{The layers of the Internet Protocol Suite}
\end{table}


In the following sections we will give a short introduction to the protocols we're
investigating in this thesis.



\subsection{\glsentrylong{coap}}

\gls{coap} is a specialized web transfer protocol designed for use with
constrained nodes and constrained networks in the Internet of Things. It is
based on the REST model, where the server makes resources available  under a
URL. Clients access these resources using the HTTP-verbs GET, PUT, POST and
DELETE. Designed to use minimal resources, both on the device and on the
network.

\begin{itemize}
    \item Application level protocol.
    \item Can carry any data format.
    \item UDP on IP.
    \item Standardized in RFC 7252.
    \item Simple binary base header format
\end{itemize}


\subsection{\glsentrylong{amqp}}

\gls{amqp} is a messaging middleware that can utilize different transport
protocols.

\begin{itemize}
    \item Support both request/response and publish/subscribe communication
    paradigms.
    \item Reliable when facing network disruptions, since it employs a
    broker-based architecture with store-and-forward capabilities.
\end{itemize}

\subsection{\glsentrylong{mqtt}}

MQTT is a client server publish/subscribe messaging transport protocol
\cite{oasis-mqtt}. It is considered lightweight and is designed for use in networks
where the bandwidth is limited. It is broker-based, where the broker is
responsible for delivering messages to clients based on the topic of a
message.

\subsection{\glsentrylong{sctp}}

\gls{sctp} is transport-layer protocol which offers functionality from both \gls{udp} and \gls{tcp}.
\begin{itemize}
    \item Message-oriented like UDP.
    \item Ensure reliable, in sequence transport of messages with congestion control like TCP.
    \item Multi-homing and multi-streaming.
\end{itemize}


\subsection{\glsentrylong{tcp}}
\gls{tcp} is one of the core transport protocol of the Internet Protocol Suite.
\begin{itemize}
    \item Connection-oriented.
    \item End-to-end reliability.
\end{itemize}

\subsection{UDP}
WSReliability? For rest må det da bygges inn støtte i proxien.
Reliable UDP. Implmentasjon av UDP som er reliable. En gammel protokoll?
\begin{itemize}
    \item No mechanisms for flow control, packet ordering or integrity of
    messages.
\end{itemize}

\section{Proxies}

A proxy is an application which acts as an intermediary between an client and a
server. Proxies are widely in use and their usage and type varies. Example of
usage is load balancing, caching and security. Web proxies are proxies that
forward HTTP requests, which is what we are investigating in this thesis.

In this section we will briefly present available popular web proxies.

\subsection{Apache}

\subsection{Squid}

\subsection{Apache Camel}

\subsection{Nginx}


\section{Summary}


%%%%%%%%%%%% Transport protokoller %%%%%%%%%%%%%%%%%%%
\begin{table}[h]
\begin{tabularx}{\textwidth}{| X | X |}
\hline
  \textbf{Protocol} & \textbf{Summary} \\ \hline
  HTTP over TCP & Widely used. Breaks down in DIL environments.\\ \hline
  CoAP & Application level protocol designed for use in the Internet of Things. \\ \hline
  AMQP & Messaging middleware with store-and-forward capabilities.\\ \hline
  MQTT & Summary here\\ \hline
  SCTP & Similar to UDP but also provide reliable, in sequence transport of messages like TCP. \\ \hline
  TCP & The well-known transport protocol. \\ \hline
  UDP & Lacks reliability, but frameworks exist that provides it. \\ \hline
\end{tabularx}
\caption{Protocols of Interest}
\end{table}
