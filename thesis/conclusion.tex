\chapter{Conclusion and Future Work}
\label{chapter:conclusion}

In this chapter, we conclude the thesis and suggest potential future work.

\section{Conclusion}

The goal of this thesis was to investigate different ways to improve the
performance of Web services in networks characterized by unreliable connects,
high error rates, and low data rate. Web services enable interoperability between
systems, but adapting Web services meant for civilian use into limited networks
may not be feasible due to the network limitations. To adapt standard Web
services into DIL networks without requiring incorporating proprietary
solutions, based on previous research we introduced the usage of proxies. The
proxy applies optimization techniques to facilitate the usage and to increase
the performance of Web services in DIL. As a part of the thesis, we specified a
set of requirements for the proxy and implemented it using the Apache Camel
framework.\Cref{table-conclusion-requirements} lists the premises and
requirements.

% Please add the following required packages to your document preamble:
% \usepackage[normalem]{ulem}
% \useunder{\uline}{\ul}{}
\begin{table}[h]
\centering

\begin{tabularx}{\textwidth}{|l|X|}
\hline
Premise 1     & Support HTTP RESTful and W3C Web services.                                         \\ \hline
Premise 2     & Work in DIL networks.                                                              \\ \hline
Premise 3     & Be interoperable with standards-based COTS solutions.                              \\ \hline
Premise 4     & Work with security mechanisms.                                                     \\ \hline
Requirement 1  & Receive and forward HTTP requests.                                                 \\ \hline
Requirement 2  & Retain HTTP request and response headers.                                          \\ \hline
Requirement 3  & Support GZIP compression of payload.                                               \\ \hline
Requirement 4  & Handle frequent network disruptions.                                               \\ \hline
Requirement 5  & Handle disconnects over longer periods of time.                                    \\ \hline
Requirement 6  & Handle low data rates, high delays and high packet error rates.                    \\ \hline
Requirement 7  & Allow for configuration of redelivery delay and maximal number of retransmissions. \\ \hline
Requirement 8  & Support usage of different transport protocols between the proxies.                \\ \hline
Requirement 9  & Easy configuration of which protocol to use.                                       \\ \hline
Requirement 10 & Be easily extendable to include other protocols and other optimization techniques. \\ \hline
\end{tabularx}
\caption{Premises and requirements}
\label{table-conclusion-requirements}
\end{table}

In our evaluation, we tested whether our proxy solution fulfilled these premises
and the more detailed requirements we specified in
\cref{section:requirements-summary}. Through the function testing, we were able
to prove that the proxy worked with a test set of Web service applications. The
Web services successfully forwarded the requests through a deployed proxy pair,
without requiring modifications except configuring the usage of proxies. This
fulfilled premises one and three, as well as requirement 1 and 2.

Furthermore, we tested how the proxies facilitated the usage of Web services in
DIL networks. We verified that the proxy was able to overcome the disconnect
aspect of DIL by implementing a redelivery mechanism. This fulfilled requirement
5 and the disconnect aspect of premise 2. Requirement 4 regarding frequent
network disruptions was not explicitly tested, but should be achieved by design
since the proxy employs reliable protocols and an application layer redelivery
mechanism.

Requirement 3 was fulfilled by implementing GZIP compression on messages between
proxies. Optimization was identified to yield a significant performance increase with
regards the RTT time perceived by Web service clients.

We also validated that the proxy could overcome the limited aspect of DIL as the
test cases were successful in all emulated DIL networks. This fulfilled premises
2 and requirement 6. Furthermore, we supported a set of transport protocols as
the means of transporting data between proxies and with that fulfilled
requirement 8. We saw how different transport protocol affected the performance
of Web services. In most of the various types of DIL environments, using
HTTP/TCP as the inter-proxy protocol was identified as the best transport.
However, we saw that in cases where the message payload was low and in networks
with low data rates, using a \gls{coap} proxy was the best option. We also
discovered how the Californium implementation of CoAP with default configuration
caused a sub-optimal utilization of an Ethernet link. Tuning the block-size
configuration could improve the CoAP's proxy performance also for larger
payloads.

Next, we were able to show that the proxy works with security mechanisms through
a proof-of-concept test using secure Web services.

Finally, the proxy implements a configuration setup that allows the user to
specify different properties of the proxy. A user of the proxy can easily
configure properties of the redelivery mechanism and change the transport
protocol used between proxies. The proxy has been designed to be easily
extendable to include other protocols and optimization techniques. The Apache
Camel framework used in the implementation facilitates this by supporting a
component based transport mechanism, as well as easily allowing customization of
the payload. This satisfies requirement 7, 9 and 10.


All in all, the goal of the thesis was reached. We developed a prototype proxy
and gave a recommendation regarding optimization techniques for Web services in
DIL environments. \Cref{table:conclusion-recommendations} summarize our
recommendations.

\begin{table}[h]
\begin{tabularx}{\textwidth}{| l | X |}
    \hline
  \textbf{DIL Network} & \textbf{Optimization recommendation} \\ \hline
  \gls{satcom} & HTTP proxy with GZIP \\ \hline
  \gls{los} & HTTP proxy with GZIP  \\ \hline
  WiFi 1 & HTTP proxy with GZIP  \\ \hline
  WiFi 2 & HTTP proxy with GZIP  \\ \hline
  \gls{cnr} & CoAP proxy with GZIP  \\ \hline
  Edge & HTTP proxy with GZIP \\ \hline
\end{tabularx}
\caption{Optimization recommendations for DIL networks}
\label{table:conclusion-recommendations}
\end{table}

Further possible investigations in this area and improvements
to the proxy are discussed in the next section.


\section{Future Work}
IPSEC.

\subsection{Andre protokoller}
SCTP, OPC-UA

\subsection{Improvements of Proxy}

Runtime valg av optimalisering.


%Not necessary to use proxy message here, since we can simply forward the HTTP request.


\subsection{Known bugs}

Some HTTP Request headers are also included in the HTTP response.

\subsubsection{HTTP}
