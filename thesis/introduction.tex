
\chapter{Introduction}

Today the Internet connects millions of users from all over the world. It plays
an important role for both businesses and people in their everyday life by
enabling the possibility to access and exchange information. The communication
infrastructure in today's society provides fast and stable access to the
Internet. However, this infrastructure might not be present in all use-cases
requiring the exchange of information over the Internet. Consider for example a
nature disaster damaging the communication infrastructure, which limits the
quality of network connections. In such a scenario the exchange of information
is critical for public health and security services. Another consideration is
the development of the \gls{iot}, where more and more devices are becoming
connected to the Internet. Typical IoT devices are sensors with limited energy
and wirelessly connected to the Internet. High packet loss rates, low data
rates, and instability may characterize such wireless networks. They are often
referred to as Low-Power and Lossy Networks (LLNs). In this thesis, we look into
improving the performance of Web services operating in these types of
conditions, with a focus on military application.


Military units may operate in areas where conditions like terrain, obstacles,
and radio interference make communication difficult. They may operate far from
existing communication infrastructure and rely only on wireless communication.
Such communication is often characterized by unreliable connections with low
data rate, long delays, and high error rates. In a military scenario, it is
necessary for units to exchange information seamlessly across different types of
communication systems. This ranges from remote combat units at the tactical
level, to commanding officers at the operational level in a static headquarters
packed with computer support. To the \gls{nato}, this concept is referred to as
\gls{nec}. In a feasibility study, \gls{nato} identified the \gls{soa} paradigm
and the Web Service technology as key enablers for information exchange in
\gls{nato} \cite{nnec-study}.

Web service technology is well tested and in widespread use in civil
applications where the network is stable and the data rate is abundant. However,
military communication may suffer from limited networks, which can leave Web
services built for civilian use unusable. How to overcome these challenges are
investigated in this thesis. The primary approach looks into how using
alternative network transport protocols may increase speed and reliability.

\section{Background and Motivation}

\gls{nato} is a military alliance consisting of 28 member countries
\cite{nato-homepage-member-countries} and which primary goal is to protect the
freedom and security of its members through political and military means. In
joint military operations, the relatively large number of member countries can
be a challenge when setting up machine-to-machine information exchange.
Differences in communication systems and equipment contribute to making the
integration of such systems more difficult. To address this issue, NATO has
chosen the \gls{soa} concept, which when built using open standards facilitates
interoperability between member nations \cite{nnec-study}.

\subsection{\glsentrylong{soa}}

\gls{soa} is an architectural pattern where application components provide
services to other components over a network. \gls{soa} builds on concepts such
as object-orientation and distributed computing and aims to get a loose coupling
between clients and services. In their reference model for \gls{soa}, the
\gls{oasis} defines \gls{soa} as \cite{oasis-soa-reference-model}:

\paragraph{}

\textit{Service Oriented Architecture is a paradigm for organizing and utilizing
distributed capabilities that may be under the control of different ownership
domains. It provides a uniform means to offer, discover, interact with and use
capabilities to produce desired effects consistent with measurable preconditions
and expectations.}

\paragraph{}

In \gls{soa}, business processes are divided into smaller parts of business
logic, referred to as \textit{services}. A service can be business related, such
as a patient register service, or an infrastructure service used by other
services and not by a user application. \gls{oasis} defines a service as
\cite{oasis-soa-reference-model}:

\paragraph{}
\textit{
A service is a mechanism to enable access to one or more capabilities, where the
access is  provided using a prescribed interface and is exercised consistent
with constraints and policies as  specified by the service description.
}

\begin{figure}[h]
\includegraphics[scale=0.6]{images/SOA.pdf}
\caption{The three roles in SOA}
\label{figure-soa-roles}
\end{figure}

Services are provided by \textit{service providers} and are consumed by
\textit{service consumers} as illustrated in \cref{figure-soa-roles}. The
service provider is responsible for creating a service description, making the
service available to others and implementing the service according to the
service description. Services are provided to service consumers through a form
of \textit{service discovery}. This can be a static configuration, or more
dynamic with a central \textit{service registry}, where service providers
publish service descriptions. Service consumers find the services they need by
contacting the service registry. The communication between services occurs
through the exchange of standardized messages.

Following the \gls{soa} principles dictate a very loose coupling between
services and the consumers of those. This allows software systems to be more
flexible, as new components can be integrated with minimal impact on existing
systems. Another aspect of loose coupling is concerning time, which enables
services and its consumers not to be available in the same instance of time.
This allows asynchronous communication. Loose coupling with regards to location
permits the location of a service to be changed without needing to reprogram,
reconfigure, or restart the service consumers. This is possible through the
usage of runtime service discovery, which is the dynamic retrieval of the new
location of the service.

Furthermore, \gls{soa} enables service implementation neutrality. The
implementation of a service is completely separated from the service
description. This allows re-implementation and alteration of a service without
affecting the service consumers. Thus, this can attribute to keep development
costs low and avoiding proprietary solutions and vendor lock-in. Another
benefit with \gls{soa} is re-usability by dividing common business processes
into services, which may help cost reduction and avoids duplication.

SOA is only a pattern, and a range of technologies can realize the concepts. The
most common approach used is the \gls{w3c} Web service family of standards,
which use the SOAP messaging protocol. To achieve interoperability between
systems from different nations and vendors, NATO has chosen this technology in
order to realize the \gls{soa} principles \cite{soa-baseline}. This allows
member countries to implement their own technology as long as they adhere to the
standards. The Web service technology is discussed in detail in
\cref{w3c-web-services}. Another method to realize the \gls{soa} principles is
\gls{rest}, an architecture style which has gained a lot of traction in the
civilian industry. We discuss REST further in \cref{rest}.

Both REST and W3C Web services are in widespread use both in the civilian and
military world. However, employing Web service solutions directly for military
purposes may not be so straightforward. These technologies were not specifically
designed to handle the network conditions found in certain military
communication networks. In the following sections, we present an overview of
military networks, discuss the characteristics of them and the possible
challenges of using Web services in them.

\subsection{Military Networks}

Military communication networks are complex and consist of many different
heterogeneous network technologies. We can group them into layers, which have
different characteristics as can been seen in \cref{figure:military-networks}.
At the highest level, there is fixed infrastructure and a relatively static
number of services. At the lower levels, there are fewer services, but the units
operating at this level are much more dynamic. The lower levels are called
tactical networks and are discussed in the next paragraph.

\begin{figure}[h]
\includegraphics[scale=0.4, left]{images/Network_Complexity.pdf}
\caption{Complexity of military networks (from \cite{pervasive-web})}
\label{figure:military-networks}
\end{figure}

\subsubsection{Tactical Networks}

Tactical networks are characterized by that the military units are deployed to
operate on a battlefield. We distinguish between deployed and mobile tactical
networks, where deployed may use existing communication infrastructure. Mobile
tactical networks have no existing communication infrastructure and therefore
experience the largest communication challenges.

 In tactical networks, military units use tactical communication equipment, which
 includes technologies like VHF, UHF, HF, tactical broadband and satellites
 \cite{ist-090}. Examples of such units are mobile units like vehicles, foot
 soldiers and field headquarters. NATO studies\cite{nato-disadvantaged-grids}
 have identified tactical networks to have the following characteristics: \\

\textit{
Disadvantaged grids are characterized by low bandwidth, variable throughput,
unreliable connectivity, and energy constraints imposed by the wireless
communications grid that link the nodes.
}

\paragraph{}

These types of networks are often called disadvantaged grids or \gls{dil}
environments, which is the term we use in this thesis.

\subsection{\glsentrylong{dil} Networks}
\label{dil}

To improve the performance of Web services in tactical networks, it is important
to understand their limitations. The \gls{dil} concept refers to three
characteristics of a network: \textit{Disconnected, Intermittent} and
\textit{Limited}.

\begin{description}
\item[Disconnected]

Military units that participate in a tactical network may be highly mobile and
may disconnect from a network either voluntarily or not. Unplanned loss of
connectivity can be due to various reasons, such as loss of signal or equipment
malfunction. The disconnected term refers to that units may be disconnected for
a long time, possibly for multiple hours or even days.

\item[Intermittent]

Units operating in a \gls{dil} environment may lose connection temporarily
before reconnecting again. The duration can range from milliseconds to seconds.
As an example, consider a military vehicle that is driving on a countryside
road. It may temporarily lose connection due to the signal being obstructed by
trees beside the road, driving into tunnels or by having a bad radio signal.

\item[Limited] Limited refers to various ways a network can be constrained. The
available data rate may be low, the network delay may be high, and the \gls{per}
may be high. The term data rate refers to the speed that data that can be
transmitted over a network. Delay means the time it takes for data to travel
from machine to machine. The packet error rate refers to the percent of packets
being sent incorrectly due to the data being erroneous altered in transmission.
A packet is considered as incorrect if at least one bit error in the data
occurs. \Cref{section:network-mectrics} elaborate on these metrics.

\end{description}

In addition to network limitations, other factors may also limit communication
for military units. As an example, consider a military foot patrol that is
operating out in the field. To communicate critical information with other units
they use radios. The radio communication equipment is powered by batteries,
which the soldiers have to carry with them. Running applications and the sending
and receiving of data can consume a considerable amount of power. Thus, the
battery could be a scarce resource for the units operating in a DIL environment.
This is similar to the constraint imposed on \gls{iot} devices.


\section{Problem Statement}
\label{section:problem-statement}

The Web service technology enables interoperability between systems, but also
increases the information overhead, requiring higher data rate demands. Most of
the Web service solutions used today are aimed for civilian purposes and do not
necessarily perform well in military environments. In contrast to civilian
networks where the data rate is abundant, mobile tactical networks may suffer
from high error rates, low data rate, and long delays. Adapting Web service
solutions meant for non-military networks directly for military purposes may not
be possible. Therefore, Web services need to be adapted to handle unreliable and
limited networks. However, it can be very expensive to alter existing Web
service technology and incorporate proprietary solutions.


\section{A Suggested Approach}
\label{section:hypothesis}

The NATO research group titled "SOA Challenges for Real-Time and Disadvantaged
Grids" (IST-090) has previously investigated which improvements that could be
made to enable the usage of \gls{cots} applications in \gls{dil} networks.
\Gls{cots} is a term used to describe the purchase of standard manufactured
systems rather than custom made. The research group pointed out the desire to
optimize Web services, but without the need of incorporating proprietary and
ad-hoc solutions \cite{ist-090}. IST-090 did not find a magic bullet that would
solve all problems with using Web services in DIL networks but identified some
factors that would offer measurable improvements. The most notable findings
were:

\begin{itemize}

    \item Foundation on open-standards.

    \item Ease of management and configuration.

    \item Transparency to the user.

    \item The Web services should be optimized without the need to incorporate
    proprietary, ad hoc solutions that ensure tighter coupling between providers
    and consumers of services.

\end{itemize}

The last bullet point refers to the issue of when we have identified
optimization techniques for DIL environments, where do we apply them? One
approach could be to modify Web service applications themselves. However, this
would mean that every application deployed in a tactical network would require
modification. The alteration would require a lot of resources and severely limit
the flexibility of using standardized Web services.

IST-090 recommends another approach; applying optimizations in proxies without
altering the Web services themselves \cite{ist-090}. A proxy is a node deployed
somewhere in a network, which applications can tunnel their network traffic
through. With this approach, the only thing required to do is to configure the
applications to send and receive data through the proxies. The proxies will then
handle the optimization for tactical networks.
\Cref{figure-proposed-proxy-solution} illustrates a setup like this, where
clients can invoke Web services through a proxy pair over a DIL network. By
placing the optimization in proxies, the Web services themselves can remain
unchanged.

\begin{figure}[h] \includegraphics[width=\textwidth]{images/dil.pdf}
\caption{Proposed proxy solution} \label{figure-proposed-proxy-solution}
\end{figure}

This approach is identified by IST-090 and is explored in this thesis. Based on
this recommendation, we create a proxy with the aim of facilitating Web service
usage in DIL networks.


\section{Premises of the Thesis}

In this section, we define the premises for the thesis and the proxy being
developed as a part of it. As we have previously discussed, W3C Web services are
in widespread use in NATO. Also, the REST architectural style has been identified
as a technology of interest to NATO. As we discuss later in \cref{web-services},
\gls{http} is the by far most common transport protocol used by these types of
services. The first and second premise are therefore that the proxy must be able
to support both REST and W3C Web services deployed in a DIL network.

Next, to optimize Web services in DIL environments, the applications themselves
should not be required to be customized. All optimization should be placed in
proxies. By doing this, we retain the interoperability with standards-based COTS
solutions. The fourth and final premise are that the proxy must work with
standard security mechanisms. In our case, this means that any messages sent
through the proxies must be exactly the same at the receiver as it would have
been without the proxies. The reason for this is due to both the header fields
and the body of the message can be part of security mechanisms, such as digital
signatures and the presence of authentication header fields.

To summarize, the premises of this thesis are that the proxy solution must:

\begin{enumerate}
    \item Support HTTP RESTful and W3C Web services.
    \item Work in DIL networks.
    \item Be interoperable with standards-based COTS solutions.
    \item Work with security mechanisms.
\end{enumerate}

\section{Scope and Limitations}

The goal of this thesis is to investigate optimization techniques for Web
services in \gls{dil} environments. We limit it to techniques that can be
applied to the application or the transport layer of the Internet protocol suite
(see \cref{figure-network-layers}). The reason for this is that NATO has
previously decided that all data communication in NATO should occur with IP
packets \cite{nnec-study}. We, therefore, limit our optimization possibilities to
the mentioned layers.

We mainly focus on the performance of Web services, yet security is of paramount
importance in military networks. Hence, any optimization techniques applied
should be possible to use together with common security mechanisms. Another
aspect is that applications that are to be used in military networks need to be
approved by security authorities. If the application is too complex, e.g. it has
a very large code base or use a lot of external frameworks, the approval process
can be very lengthy. It is therefore an important consideration to make the
proxy as relatively simple as possible.


\section{Research Methodology}

Research is the systematic investigation of how to find answers to a particular
problem. It is broadly classified into \textit{Basic Research} and
\textit{Applied Research} \cite{rajasekar2006research}. Basic research, also
called fundamental or pure research, is research on basic principles and reasons
for the occurrence of a particular event or process or phenomenon. It does not
necessarily have any practical application. Applied research, on the other hand,
is concerned about solving problems employing well known and accepted theories
and principles. In this thesis, we set out to solve the actual real-world
problem of optimizing Web services. Thus, we perform applied research. To
address this problem we need a systematic approach of how to conduct the
research. This is referred to as \textit{research methodology} and says
something about how the research is to be carried out.

In our work, we are performing research in the area of Computer Science, a
scientific discipline defined as \cite{denning}: \\

\textit{The systematic study of algorithmic processes that
describe and transform information: their theory, analysis, design, efficiency,
implementation, and application}.

\paragraph{}

 \citeauthor{denning} have identified three main processes for the computer
 science discipline, \textit{theory, abstraction} and \textit{design}
 \cite{denning}. \textit{Theory} derives from the mathematics discipline and
 applies to the areas of computer science that rely on underlying mathematics.
 Examples of this are the computer science areas of algorithms and data
 structures that involve complexity and graph theory. The next process,
 \textit{abstraction}, deals with modeling potential implementations. The
 \textit{design} process is the process of specifying a problem, transforming
 the problem statement into a design specification, and repeatedly inventing and
 investigating alternative solutions until a reliable, maintainable, documented,
 and tested design is achieved.

The research methodology used in this thesis is based on the design process. The
four steps and the efforts undertaken in them are summarized here:

\begin{description}

    \item[Specify the problem] The primary focus of this thesis is how to
    improve the performance of Web services in DIL environments. We formulate a
    problem statement in \cref{section:problem-statement} and propose a possible
    solution in \cref{section:hypothesis}. Moreover, we present the technical
    background in \cref{chapter:background} and previous related work in this
    area in  \cref{chapter:related-work}.

    \item[Derive a design specification based on the requirements] Based on the
    premises, scope of the thesis, studies of the technological background, and
    related work, we derive a set of requirements and specifications in
    \cref{chapter:requirements}.

    \item[Design and implement the system] After we identify the requirements
    for the proxy solution, we design and implement them.
    \Cref{chapter:design} elaborate this step.

    \item[Evaluate the system] Finally, the solution is assessed through a
    series of tests. The purpose of this is to evaluate if we are in fact able
    to solve the problem we set out to solve. We cover the testing and
    evaluation in \cref{chapter:evaluation} and draw a conclusion in
    \cref{chapter:conclusion}.

\end{description}


\section{Contribution}

The outcome of this thesis is a recommendation regarding which optimizations
techniques can be used in DIL networks to increase the performance of Web
services. As a part of this work, we implement a prototype DIL proxy.

\section{Outline}

The remainder of this thesis is organized as follows:

Chapter 2 presents the technical background for this thesis. We introduce
computer networks in general before we dive into different communication
paradigms and protocols. Then, in chapter 3, we discuss previous work done in
the area. In chapter 4, we derive a specification for the proxy, before we in
chapter 5 present the design and implementation details. Next, in chapter 6 we
show the testing of the proxy solution and how the solution fulfilled the
premises and requirements. Finally, in chapter 7, we summarize our work and
provide reflections on possible future work within this field.

In the appendix, we include information about the test results, proxy
configuration and the source code developed as a part of this thesis.
