\documentclass[USenglish]{ifimaster}

\usepackage[utf8]{inputenc}
\usepackage{ifikompendiumforside}
\usepackage{graphicx}
\usepackage[acronym]{glossaries}
\usepackage[backend=biber,sorting=none]{biblatex}

\usepackage{tabularx}
\usepackage{cleveref}

\makeglossaries

% Acronyms
\newacronym{dil}{DIL}{Disconnected, Intermittent and Limited environments}

\title{Improving the performance of Web Services in Disconnected, Intermittent and Limited Environments}
\author{Joakim Johanson Lindquister}
\bibliography{references}

\begin{document}
\ififorside{}

\chapter*{Abstract}
In this thesis I investigate different techniques to improve the performance of Web services in typical tactical network environments.
\pagebreak

\tableofcontents
\listoftables
\listoffigures

\pagebreak


\chapter{Introduction}
Military units may operate under conditions where the reliability of the network connection is low. They can operate far from existing communication infrastructure and rely only on wireless communication. Such networks are often characterized by unreliable connections with low bandwidth and high error rates making data communication difficult. In a military scenario it is necessary for units at all operational levels to seamlessly exchange information across different types of communication systems. To NATO, this concept is referred to as Network Enabled Capability(NEC). In a feasibility study, NATO identified the Service-Oriented Architecture and Web Services as key enablers\cite{nnec-study}.

Web services is well tested in civil environments where the network is stable and the bandwidth is abundant. However, military tactical networks may suffer from high error rates and low bandwidth which can leave the Web services unusable. To overcome these challenges we can apply optimization techniques at different layers of the protocol stack which is investigated in this thesis.


\section{Background and Motivation}
NATO is an military alliance consisting of 28 member countries\cite{nato-homepage-member-countries} and which primary goal is to protect the freedom and security of its members through political and military means. In joint military operations the relatively large number of member countries can be a challenge when setting up machine-to machine information exchange. Differences in communication systems and equipment can attribute to make data communicating difficult. In order to address this issue, NATO has chosen Service-Oriented Architecture concept.

\subsection{Service-Oriented Architecture}
Service-Oriented Architecture(SOA) is an architectural pattern where application components provide services to other components over a network. SOA is built on principles such as object-orientation and distributed computing and aims to get a loose coupling between clients and services. In their reference model for Service Oriented Architecture\cite{oasis-soa-reference-model}, the Organization for the Advancement of Structured Information Standards (OASIS) defines SOA as:
\paragraph{}
\textit{Service Oriented Architecture is a paradigm for organizing and utilizing distributed capabilities that may be under the control of different ownership domains. It provides a uniform means to offer, discover, interact with and use capabilities to produce desired effects consistent with measurable preconditions and expectations.}

\paragraph{}
In SOA business logic is divided into smaller chunks of logic, referred to as \textit{services}. A service can be business related, e.g a patient register service, or a infrastructure service used by other services and not by a user application.  Services are provided by \textit{service providers} and are consumed by \textit{service consumers}. The service provider is responsible for creating a service description, making the service available to others and implementing the service according to the service description. Services are made available through a \textit{service registry}, where service providers publish service descriptions. Service consumers finds the services it needs by contacting the service registry. The communication between services occur through the exchange of standardized messages.

Following the SOA principles dictates a loose coupling between services and service consumers which allows software system to be more flexible. Loose coupling with regards to time enables services and its consumers to not be available at the same instance of time. This enables asynchronous communication. Loose coupling with regards to location allows the location of an service to be changed without needing to reprogram, reconfigure, or restart the service consumers. This is possible through the usage of the service registry, which is updated with the new location.

Furthermore SOA enables service implementation neutrality. The implementation of service is completely separated from the service description. This allows re-implementation and alteration of a service without affecting the service consumers. Thus this can attribute to keep development costs low and avoiding proprietary solutions and vendor lock-in. Another benefit with SOA is reusability by dividing common business processes into services, which may help cost reduction and avoids duplication.

SOA is only a pattern and the concepts can be realized by a range of technologies. The most common used approach is the Web service standard, using the SOAP messaging protocol. To achieve interoperability between systems from different nations and vendors, NATO has chosen the Web service technology in order to realize the SOA principles. This allows member nations to implement their own technology as long as they adhere to the standards. Web services are are discussed in detail in \cref{web-services}.

\subsection{Tactical networks}
Web services can be used in strategic military networks as they have network infrastructure with the same characteristics as civil networks. However, mobile tactical networks are characterized by that the units use tactical communication equipment which includes technologies like VHF, UHF, HF, tactical broadband and satellites. Examples of such units are mobile units like vehicles, foot soldiers and field headquarters. These types of networks have low bandwidth, possibly high delay, high error rates and frequent disconnections. They are often called disadvantaged grids or DIL. NATO studies has identified such networks to have the following characteristics:

\paragraph{}
\textit{Disadvantaged grids are characterized by low bandwidth,variable throughput, unreliable connectivity, and energy constraints imposed by the wireless communications grid that link the nodes}\cite{nato-disadvantaged-grids}.

\paragraph{}
The characteristics of these networks and what challenges they impose are discussed in detail in \cref{dil}.

\section{Problem Statement}
Most of the Web Service solutions used today are aimed for civilian use and does not necessarily perform well in military environments. In contrast to civilian networks where bandwidth are abundant, mobile tactical networks may suffer from high error rates and low bandwidth. Adapting Web service solutions meant for civil networks directly for military purposes may not be possible. Therefore, Web services needs to be adapted in order to handle network challenges. However, it can be very expensive to alter existing Web service technology and build own proprietary solutions. It is much better to use commercial off-the-shelf software. By placing the optimization in proxies, the Web services can remain unchanged.

In my master thesis I will investigate different optimization techniques that can be applied in order to improve Web service performance in disadvantaged networks. In order for the clients and services to remain interoperable the optimization techniques will be placed in proxies. The Web Services will communicate as normal, while all network traffic is tunneled through a proxy. The Web Service itself does not need to pay attention to the bad connectivity, the proxy will choose the appropriate protocol and configuration.

\section{Premises}
The Web services does not need to be changed.

\section{Scope and Limitations}
Snevre inn oppgaven.

\section{Research Methodology}
Denning.

\section{Contribution}
The outcome of this thesis is an recommandation regarding which optimizations techniques which can be used in DIL to enhance the performance of Web services.

\section{Outline}
Hvordan er resten av oppgaven strukturert.


%%%%% BACKGROUND %%%%
\chapter{Background}
In this part, I will present relevant technologies.
\section{Related Work}
Diskuterer eksisterende arbeid.



%%%%%%% WEB SERVICES
\section{Web services}
\label{web-services}
Web services are client and server applications that communicate over the World Wide Web and can be used to implement a service-oriented architecture. There exists many definitions of Web services where the core principles are the same, but the finer details may vary. The World Wide Web Consortium has defined Web Services as\cite{wrc-web-service}:
\paragraph{}
\textit{A Web service is a software system designed to support interoperable machine-to-machine interaction over a network. It has an interface described in a machine-processable format (specifically WSDL). Other systems interact with the Web service in a manner prescribed by its description using SOAP-messages, typically conveyed using HTTP with an XML serialization in conjunction with other Web-related standards.}

\paragraph{}
This definition points out a set of standards that enables machine-to-machine interactions. These standards are discussed in the following sections.

Figur her.

\subsection{XML}
Extensible Markup Language(XML) is a markup language and is considered as the base standard for Web services. An XML document consist of data surrounded by tags and is designed to be both machine and user readable. In a Web service, XML is used to tag the data.

\subsection{Service descriptions: WSDL}
Web Services Description Language(WSDL) is an interface definition language that using XML describes functionality offered by a Web service. The interface describes available functions, data types for message requests and responses and binding information about the transport protocol, as well as address information for locating the service.

\subsection{SOAP}
SOAP is an application level XML based protocol specification for information exchange in the implementation of Web services. It is transport protocol agnostic and can be carried over various protocols. The far most used is HTTP over TCP, but other protocols such as UDP and SMTP can be used as well.

A SOAP message consist of a optional header and a required body. The header can contain information not directly related to the message, such as routing information for the message. The body contains the data being sent, known as the payload.

\subsection{Other Web services}
However, there also exist other types of Web services which does not follow the previously discussed standards. RESTful web services let users manipulate data using a set of stateless operations. It uses exclusively HTTP. REST services are easy to understand and have gained a lot of traction in the civil industry in the latest years.



%%%%%%%%%% DIL %%%%%%%%%%%%%%%
\section{DIL}
\label{dil}
\paragraph{}
These constrains of mobile tactical networks are central in order to understand the problem at hand, and I will therefore explain the concepts here:

\begin{description}
\item[Bandwidth and throughput] The terms bandwidth and throughput are used interchangeably in the networking community and refers to the data transfer rate; how fast data can be transported from one point to another in given time period. This is often expressed in bits per second.
\item[Unreliable connectivity] Units that are participating in a tactical network are highly mobile and may disconnect from a network either voluntarily or not. Unplanned loss of connectivity can be due to various reasons, such as loss of signal or equipment malfunction.
\item[Energy constraints imposed by the wireless communication grid] The battery capacity and the transmission range of the communication equipment for mobile units may be limited. Another issue is that in some cases military units are required to enter radio silence in order to avoid being detected by the enemy. During radio silence units may only receive data and not send any.
\end{description}

Theese constrains imposes some challenges when employing Web services in tactical networks. In paper X, tree areas that need to be addresses are identified\cite{IST-118}.

\label{section:DIL-problems}
\subsubsection{End-to-end connections}
Attempting to establish and maintaining connections in DIL environments can lead to increased communication overhead and possible complete breakdown of communication.
\subsubsection{Network heterogeneity}
Different technologies used. Different performance in networks may lead to buildup of data in buffers, risking loss of information.
\subsubsection{Web Service overhead}
Web services generates overhead.
\section{Optimization techniques}
Web services enables interoperability betweens system, but also increase the information overhead, requiring higher data rate demands. By using proxies, we can freely choose the communications protocols and configurations between the proxy pair without altering the Web Services themselves. In this thesis I will investigate different techniques in order to optimize the communication between a Web Service and a Web Service client. The optimization can be implemented at different levels of the protocol stack in use. \Cref{table:optimalization-overview} lists an overview of possible optimization techniques studied in this thesis.
\\ \\

%%%%%%%% TABLE OPTIMALIZATION OVERVIEW %%%%%%%%%%%%%
\begin{table}[h]
\begin{tabularx}{\textwidth}{| X | X |}
\hline
  \textbf{Protocol Stack} & \textbf{optimization possibilities} \\ \hline
  The application & Optimize the application\\ \hline
  Web service messaging: SOAP & Optimize SOAP, e.g XML compression \\ \hline
  HTTP/TCP, UDP or other transport protocols & SOAP is transport agnostic. Other protocols can be used. \\ \hline
  IP & NATO NEC feasibility study suggests that all protocols should be over IP. \\ \hline
  Lower layers & Not in the scope of this thesis. \\ \hline
\end{tabularx}
\caption{optimization possibilities.} \label{table:optimalization-overview}
\end{table}


\subsection{Compressing the payload}
The first optimization techniques deals with the optimization of the encoding. By compressing the Web service payload, we can reduce the amount of data that need to be sent.
\begin{itemize}
\item GZIP
\item EFX(Efficient XML).
\end{itemize}
Previous experiments shows EFX has the compression results with GZIP as the second best alternative\cite{johnsen-trude-compression-techniqes}.

\subsection{Reducing overhead of SOAP}
HTTP/TCP is the most used transport protocol for SOAP messages, but since SOAP is transport protocol agnostic different protocols can be used. Experiments show that this is possible.


%%%%%%%%%%%% Transport protokoller %%%%%%%%%%%%%%%%%%%
\begin{table}[h]
\begin{tabularx}{\textwidth}{| X | X |}
\hline
  \textbf{Transport Protocol} & \textbf{Summary} \\ \hline
  HTTP over TCP & Widely used. Breaks down in DIL environments.\\ \hline
  Military Message Handling System(MMHS) & Optimize the application\\ \hline
  Stream Control Transmission Protocol(SCTP) & Features multihoming. \\ \hline
  Advanced Message Queuing Protocol(AMPQ) & Application level protocol. Employs a broker architecture with store-and-forward capabilities. \\ \hline
  SOAP directly on TCP & Its possible \\ \hline
\end{tabularx}
\caption{Transport protocols}
\end{table}

\section{Proxies}
A proxy is a server that acts as intermediary for requests between a client and server over a network. Proxies can be used to accomplish different tasks, from caching web content to bypassing filtering and censorship. We can group proxies into three types, forwarding proxies, reverse proxies and gateway proxies.

\section{Requirement Analysis}
A discussed in \cref{section:DIL-problems}, the dependency on end-to-end connections needs to be removed. This can be done by adding proxies to the network.
Mobile units have to carry batteries with them and the capacity is therefore limited. Advanced compression techniques may reduce the overhead, but also requires more battery. This trade-off needs to be considered.
\\ \\ \\
\begin{table}[h]
\begin{tabular}{| l | l |}
\hline
  \textbf{Requirement} & \textbf{Priority} \\ \hline
  Receive and forward HTTP 1.X requests & 1\\ \hline
  Allow modifications on the payload & 1 \\ \hline
  Allow configuration of HTTP timeouts & 1 \\ \hline
  Keep HTTP-connection alive & 1 \\ \hline
  Support protocol X and y & 2 \\ \hline
\end{tabular}
\caption{Proxy requirements}
\end{table}

\section{Summary}



%%%% Design and implementation %%%%
\chapter{Design and Implementation}
\section{Overall Design}
\section{Proxy}
\subsection{Squid}
Squid is a fully-featured HTTP/1.0 proxy.
\begin{figure}[h]
\includegraphics[scale=0.4]{images/architecture.png}
\caption{Architectural overview of proposed design}
\end{figure}



\subsection{Tuning application server configuration}

\subsection{Alternative transport protocols}

\section{Summary}

\chapter{Testing and Evaluation}
\section{Evaluation Tools}

\chapter{Conclusion and Future Work}
\section{Conclusion}

\section{Future Work}

\pagebreak
\printbibliography{}
\printglossary

\end{document}
