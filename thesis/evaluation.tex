\chapter{Testing and Evaluation}

To answer our research question of how to improve the performance of Web
services in dil environments, we developed some test scenarios. In this chapter
we present how the testing was performed and then present the results we
obtained.

\section{Test Scenarios}
\begin{enumerate}
    \item REST-client and server sending each other GET requests.
    \item W3c Web services.
\end{enumerate}


\section{\glsentrylong{netem}}

In order to simulate \gls{dil} environments we need some way to control the
properties of the network traffic. Fortunately, the Linux kernel offers a rich
set of tools for managing and manipulating the transmission of packets.

%Siter tldp -> Traffic Control HOWTO

\gls{netem} is an enchancement of the traffic control facilities that allows us
to control delay, packet loss and other characteristics to packets outgoing from
a selected network interface.
%Siter man-page om tc-netem

\subsection{NetEm emulating}

\textbf{tc}(traffic control) is a linux program to configure and control the
linux kernels Network scheduler.

\subsubsection{Delays}

NetEm can emulate delays on packets on a specific link.

\begin{lstlisting}[frame=single, caption="Emulating delay"]
  tc qdisc add dev eth0 root netem delay 100ms
\end{lstlisting}

In this example we add a fixed delay on 100 ms to all packets going out of local
Ethernet.

\section{Execution of tests}

Although testing on regular machines gives us a indication, to get as realistic
results as possible we also performed tests on military communication equipment.

\subsection{Testing on computers}
Describe test setup.

\subsection{Testing on real hardware}
Kongsberg radio.


\section{Evaluation Tools}
