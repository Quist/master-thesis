\chapter{Testing and Evaluation}

In this chapter we present how the testing was performed and the present the
results we obtained.

\section{\glsentrylong{netem}}

In order to simulate \gls{dil} environments we need some way to control the
properties of the network traffic. Fortunately, the Linux kernel offers a rich
set of tools for managing and manipulating the transmission of packets.

%Siter tldp -> Traffic Control HOWTO

\gls{netem} is an enchancement of the traffic control facilities that allows us
to control delay, packet loss and other characteristics to packets outgoing from
a selected network interface.
%Siter man-page om tc-netem

\subsection{NetEm emulating}

\textbf{tc}(traffic control) is a linux program to configure and control the
linux kernels Network scheduler. 

\subsubsection{Delays}

NetEm can emulate delays on packets on a specific link.

\begin{lstlisting}[frame=single, caption="Emulating delay"]
  tc qdisc add dev eth0 root netem delay 100ms
\end{lstlisting}

In this example we add a fixed delay on 100 ms to all packets going out of local
Ethernet.


\section{Testing on real hardware}
Kongsberg radio


\section{Evaluation Tools}
