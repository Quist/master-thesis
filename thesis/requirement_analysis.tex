\chapter{Requirement Analysis}
\label{chapter:requirements}

In this chapter we discuss the requirements for optimization techniques aiming
at enabling Web services in DIL environments. These requirements build on the
scope and premises discussed in the introduction. To recap, the defining
premises were that the proxy should:

\begin{enumerate}
    \item Support HTTP RESTful and W3C Web services.
    \item Work in DIL networks.
    \item Be interoperable with standards-based COTS solutions.
    \item Work with security mechanisms.
\end{enumerate}

Based on previous research, in particular the work of the \gls{nato} research
groups IST-090 and IST-118, we are in thesis developing a proxy solution
supporting these premises. In the following sections we discuss the specific
requirements for this approach.

\section{HTTP Proxy}

The first premise implies that our proxy must accept \gls{http}, as this is the
far most used Web service transport protocol. Furthermore, the third and fourth
premises have some important implications for our proxy. Our proxy must be able
to accept HTTP requests from a Web service, forward it to the other proxy, which
in turn delivers it to the intended receiver. The communication between the
proxies is not required to be with HTTP, but rather using a protocol that deals
with DIL networks in a better way. However, since ultimately a HTTP request
should be delivered to the intended receiver, the HTTP properties must be
retained. This means that the proxy must preserve the HTTP method and headers.
Also, since REST is payload agnostic, the proxy must be able to support
different types of data being sent through it (XML, JSON etc.).

Furthermore, the proxy must be able to handle the difficult network conditions
of DIL. The specific requirements are outlined in the following sections.

\section{Cope with DIL Networks}

The \gls{dil} term refers to three aspects of a network, \textit{disconnected,
intermittent} and \textit{limited} The proxy should be able to
overcome the implications of these aspects. In the following sections we discuss
the requirements each aspect implies.

\subsection{Disconnected}

The Disconnected aspect of DIL refers to disconnects for a longer period of
time. As we saw in the previous chapter, earlier work has identified the removal
of end-to-end dependencies as important to overcome this aspect. Without
proxies, a disconnect for a longer period of time would cause a timeout
exception at the Web service, leaving it up to Web service itself to deal with
the exception. By employing a proxy pair, the end-to-end dependency is instead
moved from between a client and a Web service, and to between the client and the
locally deployed proxy. As a result, the connection between the proxies over a
DIL network can be lost, while still maintaining the connection between the
client and local proxy. When the connection is reestablished, the proxy must be
able to continue transmission of messages on behalf of clients.

This requires the proxy to have some sort of redelivery mechanism. When a proxy
detects that it unable to transmit messages to the other proxy, it should
ideally wait until the connection is reestablished before trying to send more
messages. However, the only way to know if the connection is reestablished is to
try and send more messages and see if they succeed. The first, and maybe naive
approach, could be to just retransmit the message again and again. But by doing
this, we could risk overflowing a slow receiver, as well as causing congestion
in a possibly overloaded network. Different types of networks and different use
cases for the applications involved may require different redelivery mechanisms.
At deployment, the proxy should therefore support a  set of configurable
redelivery mechanism properties:

\begin{description}

    \item[Redelivery Delay] The proxy should support the retransmission of
    sending messages with a fixed delay between each attempt.

    \item[Exponential Backoff] If exponential backoff is configured, the proxy
    should gradually try resending more and more seldom.

    \item[Maximum Redeliveries] The proxy should support user configuration of
    how many times a retransmission should be attempted before giving up.

\end{description}


\subsection{Intermittent}

The proxy should handle brief, temporary disconnects that can occur in a DIL
network. It is comparable to the disconnect aspect, as intermittent refers to a
shorter disconnect. A "long" intermittent disconnect triggers a timeout at the
application layer and should be dealt with by the proxy retransmission
mechanisms. With shorter intermittent disconnects, the transport protocol should
be able to deal with it. This requires using a reliable transport protocol, or
handling it in the application layer.

\subsection{Limited}

Limited refers to different ways a network can be limited. Accordingly, the
proxy must cope with very low data rates, possible high error rates and long
delays. This implies that reducing Web service overhead in order to lower the
amount of bytes that need to be sent over a limited network is important.
Moreover, the proxy may run on machines with restricted resources (battery
capacity), which means that a low CPU overhead is desired.

\section{Support Optimization Techniques}

To improve the performance of Web services in DIL environments, the proxy should
support a set of optimization techniques. As we discussed in the related works
chapter, there exist many approaches to optimizing Web services. Reducing Web
service overhead by using compression was identified as a technique that yields
a significant improvement. Another approach was the usage of alternative
transport protocols. In this thesis we focus on compression and the usage of
alternative protocols as the means of optimizing Web services.

\subsection{Compression}

Compression reduces the size of a message sent over a network. In order to
perform compression the proxy must be able to modify the payload of the message.
Due to security mechanisms that detect changes to the payload (digital
signatures), the payload must be restored back to its original form before being
forwarded to the final receiver. One of our premises is that we must support
both RESTful and W3C Web services. RESTful services do not put any restrictions
on the data format of a message. Thus, we cannot use XML-specific compression,
but rather we need to use general-purpose techniques.

Based on previous work we identify GZIP as the best approach for general
purpose compression.

\subsection{Proxy Protocol Communication}

One of the optimization techniques we identified is the usage of alternative
transport protocols between the proxy pair. We introduced a set of protocols in
the technical background chapter and discussed previous evaluations using them
in DIL networks in last chapter. In the following paragraphs we analyze them for
usage in the context of proxy communication in a DIL network.

\begin{description}

    \item[\gls{http}] The by far most used protocol for Web services is HTTP
    over TCP. TCP is an old and proven protocol and was originally designed to
    provide reliable end-to-end communication over unreliable networks. The
    less intrusive optimization technique would therefore be that the proxies
    simply forward HTTP-requests without using an alternative protocol.
    Although proxing Web service requests through proxies would cause some overhead from
    processing time and custom proxy headers, we still get the benefit of
    breaking the end-to-end dependency and the possibility of using compression.
    Furthermore, using HTTP allows us to compare the "standard" protocol against
    other protocols. We therefore recommend HTTP as a possible proxy pair
    communication method.

    \item[\gls{udp}] UDP has less overhead than TCP, but lacks mechanisms for
    reliability and congestion control. The lack of reliability could be handled
    at the application level instead, but would require a library on top of it.
    Furthermore, UDP is not TCP-friendly. For these reasons, we conclude that UDP
    is unfit for proxy communication as part of this thesis.

	\item[\gls{coap}] CoAP is a relatively new protocol intended for use in the
	Internet of Things. It is designed to have low overhead, low code footprint
	and be easily mapped to and from HTTP. These properties make the protocol
	very interesting as the means of communication between a proxy pair.

	\item[\gls{amqp}] AMQP is in widespread use and offers reliable message
	transmission. It supports both the request-response and publish-subscribe
	message paradigms. We therefore recommend AMQP as a possible proxy pair
	communication method.

	\item[MQTT] MQTT is a publish-subscribe messaging protocol and is considered
	as lightweight and simple to implement. However, the inter-proxy
	communication requires a request-response type of messaging. MQTT does not
	facilitate this type of communication. With that said, it is possible to
	have a request-response paradigm on top of publish-subscribe by organizing
	queues and by using some application logic. However, since MQTT does not
	natively support request-response, we do not recommend this protocol for
	proxy pair communication.

	\item[\gls{sctp}] SCTP offers functionality from both UDP and TCP. It is
	reliable and has been identified in previous related work as an interesting
	protocol for DIL networks. We therefore recommend it as a possible proxy
	communication method.

\end{description}

The proxy should support the identified protocols found suitable for
communication between proxies over a DIL network. The recommendations are
summarized in \cref{table:possible-proxy-protocols}. For evaluation purposes the
proxy should be easily configured of which protocol to use.

\begin{table}[h]
\begin{tabularx}{\textwidth}{| X | X |}
\hline
  \textbf{Protocol} & \textbf{Recommendation} \\ \hline
  HTTP & Yes \\ \hline
  UDP & No \\ \hline
  CoAP & Yes \\ \hline
  AMQP & Yes \\ \hline
  MQTT & No \\ \hline
  SCTP & Yes \\ \hline
\end{tabularx}
\caption{Protocols recommended as possible proxy communication protocol.}
\label{table:possible-proxy-protocols}
\end{table}



\section{Summary}
\label{section:requirements-summary}

In this chapter we have discussed the requirements for our proxy, which we
summarize here:

\begin{enumerate}
    \item Receive and forward HTTP requests.
    \item Retain HTTP request and response headers.
    \item Support GZIP compression of payload.
    \item Handle frequent network disruptions.
    \item Handle disconnects over longer periods of time.
    \item Handle low data rates, high delays and high packet error rates.
    \item Allow for configuration of redelivery delay and maximal number of retransmissions.
    \item Support usage of different transport protocols between the proxies.
    \item Easy configuration of which protocol to use.
    \item Be easily extendable to include other protocols and other optimization techniques.
\end{enumerate}

Next, we discuss the design and implementation of our proxy supporting the
premises and identified requirements.
