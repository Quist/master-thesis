\chapter{Requirement Analysis}
\label{chapter:requirements}

In this chapter we discuss the requirements for optimization techniques aiming
at enabling Web services in DIL environments. These requirements build on the
scope and premises discussed in the introduction. To recap, the defining
premises were:

\begin{enumerate}
    \item Support HTTP RESTful and W3C Web services.
    \item Work in DIL networks.
    \item Interoperable with standardized solutions(COTS).
    \item Work with security mechanisms.
\end{enumerate}

Based on previous research, in particular the work of the \gls{nato}
research groups IST-090 and IST-118, we are in thesis building a proxy solution
supporting these premises. In the following sections we discuss the specific requirements
for this approach.

\section{HTTP Proxy}

The first premise implies that our proxy must accept \gls{http}, as this is the far
most used Web service protocol. Furthermore, the third and forth premises have
some important important implications for our proxy. Our proxy must be able to
accept a HTTP requests from a Web service, forward it to the other proxy, which
in turn delivers it to the intended receiver. The communication between the
proxies are not required to be with HTTP, but rather a protocol that deals with DIL
networks in a better way. However, since ultimately a HTTP request should be
delivered to the intended receiver, the HTTP properties must be retained. This
means that the proxy must preserve the HTTP method and headers. Also, since
REST is payload agnostic, the proxy must be able to support different types of
data being sent through it (XML, JSON etc.).

Furthermore, the proxy must be able to handle the difficult network conditions
of DIL. The specific requirements are outlined in the following sections.

\section{Cope with DIL networks}

The \gls{dil} term refers to three aspects of a network, \textit{disconnected,
intermittent} and \textit{limited}. Any optimization techniques must be able to
handle the implications of these aspects. In the following sections we discuss
the requirements each aspect enforces.

\subsection{Disconnected}

The Disconnected aspect of DIL refers to disconnects for a longer period of
time. As we saw in the previous chapter, earlier work has identified the removal
of end-to-end dependencies as important to handle this aspect. By employing
proxies, the end-to-end dependency is instead moved from between a client and a
Web service, and to between the client and the locally deployed proxy.
Followingly, the connection between the proxies over a DIL network can still be
lost, while still maintaining the connection between the client and local proxy.
This means that the proxy must maintain the connection with the local
application, while managing loss of connection with the other proxy. When the
connection are reestablished, the proxy should continue sending the data and
finally delivering a response back to the client.

This requires the proxy to have some sort of redeliver mechanism, which after a
configurable amount of time tries to retransmit the data. If still unsuccessful,
the proxy waits an amount of time before attempting again. This can be done a
configurable amount of times, or indefinitely until the connection is
reestablished. One consideration about this is the risk of overflowing the
receiver. It is therefore common to use an exponential back off. This mechanisms
grants that the proxy tries frequently immediately after loss of connection, but
then gradually tries more and more seldom. Different use cases may require
different parameters, so back-off and redeliver delay should be able to alter
via proxy configuration.

\subsection{Intermittent}

The proxy should handle brief, temporary disconnects that can ocour in a DIL
network. It is comparable to the disconnect aspect, as intermittent refers to a
shorter disconnect. A "long" intermittent disconnect triggers a timeout at the
application layer and should be dealt with by the proxy retransmission
mechanisms. With shorter intermittent disconnects, the transport protocol should
be able to deal with it. This requires using a reliable transport protocol.

\subsection{Limited}

Limited refers to different ways a network can be limited. Accordingly, the
proxy must cope with very low data rates, possible high error rates and long
delays. This implies that reducing Web service overhead in order to lower the
amount of bytes that need to be sent over a limited network is important.
Moreover, the proxy may run on machines with restricted resources (battery
capacity), which means that a low CPU overhead is desired.

\section{Support optimization techniques}

To improve the performance of Web services in DIL environments, the proxy should
support a set of optimization techniques.

\subsection{Compression}

Compression reduces the size of a message sent over a network. In order to
perform compression the proxy must be able to modify the payload of the message.
Due to security mechanisms that detect changes to the payload (digital
signatures), the payload must be restored back to its original form before being
forwarded to the final receiver. One of our premises was that we must support
both RESTful and W3C Web services. RESTful services does not put any
restrictions on the data format of a message. Thus, we cannot use XML-specific
compression, but rather we need to use general-purpose techniques.

Based on previous work we identified GZIP as the best approach for compression.

\subsection{Proxy protocol communication}

One of the optimization techniques we have identified is the usage of
alternative transport protocols between the proxy pair. We introduced a set of
protocols in the technical background chapter and discussed previous evaluations
using them in DIL networks in last chapter. In the following paragraphs we
analyse them for usage in the context of proxy communication in a DIL network.

\begin{description}

    \item[\gls{http}] The by far most used protocol for Web services is HTTP
    over TCP. TCP is an old and proven protocol and was originally designed to
    provide reliable end-top-end communication over unreliable networks. The
    less intrusive optimization technique would therefore be that the proxies
    simply forwards HTTP-requests without using an alternative protocol. Since
    HTTP is in extensive use in many different type of networks. We therefore
    recommend HTTP as a possible proxy pair communication method.

    \item[\gls{udp}] UDP is a fast protocol with less overhead than TCP, but
    lack mechanisms for reliability and congestion control. The lack of
    reliability could be done at the application level instead, but would
    require a library on top of it. Furthermore it is not TCP-friendly. For
    these reasons we conclude that UDP is unfit for proxy communication as part
    of this thesis.

    \item[\gls{coap}] CoAP is a relatively new protocol designed for use in the
    Internet of Things. It is designed to have low overhead, low code footprint
    and be easily mapped to and from HTTP. These properties makes the protocol
    very interesting as the means of communication between a proxy pair.

    \item[\gls{amqp}] AMQP is in widespread use, and offers reliable message
    transmission. It support both the request-reply and publish-subscribe message paradigms. We therefore recommend AMQP as a possible proxy pair communication method.

    \item[MQTT] MQTT is a publish-subscribe messaging protocol and is considered
    as light weigh and simple to implement. However, due to the inter-proxy
    communication requires an request-reply type of messaging, MQTT does not
    facilitate this type of communication. With that said, it is possible to
    have a request-response paradigm on top of publish-subscribe by organizing
    queues and by using some application logic. However, since MQTT does not
    natively support request-response, we do not recommend this protocol for
    proxy pair communication.

    \item[\gls{sctp}] SCTP offers functionality from both UDP and TCP. It is
    reliable and has been identified in previous related work as a interesting
    protocol for DIL networks. We therefore recommend it as a possible proxy
    communication method.

\end{description}

The proxy should support the identified protocols suitable for communication
over a DIL network. The recommendations are summarized in
\cref{table:possible-proxy-protocols}. For evaluation purposes it should be
easily configured which protocols to use.

\begin{table}[h]
\begin{tabularx}{\textwidth}{| X | X |}
\hline
  \textbf{Protocol} & \textbf{Recommendation} \\ \hline
  HTTP & Yes \\ \hline
  UDP & No \\ \hline
  CoAP & Yes \\ \hline
  AMQP & Yes \\ \hline
  MQTT & No \\ \hline
  SCTP & Yes \\ \hline
\end{tabularx}
\caption{Protocols recommended as possible proxy communication protocol}
\label{table:possible-proxy-protocols}
\end{table}



\section{Summary}

In this chapter we have discussed the requirements for our proxy, which are
summarized here:

\begin{enumerate}
    \item Receive and forward HTTP requests
    \item Retain HTTP request and response headers.
    \item Support GZIP compression of payload.
    \item Handle frequent disconnects.
    \item Handle disconnects over longer periodes of time.
    \item Handle very low data rate.
    \item Allow for configuration of redelivery delay and maximal number of retransmissions.
    \item Support usage of different transport protocols between the proxies.
    \item Easy configuration of which protocol to use.
\end{enumerate}

Next, we discuss the design and implementation of our proxy supporting these
requirements.
