\chapter{Requirement Analysis}
As discussed in \cref{section:DIL-problems}, the dependency on end-to-end
connections needs to be removed. This can be done by adding proxies to the
network. Mobile units have to carry batteries with them and the capacity is
therefore limited. Advanced compression techniques may reduce the overhead, but
also requires more battery. This trade-off needs to be considered.
\\ \\ \\
\begin{table}[h]
\begin{tabular}{| l | l |}
\hline
  \textbf{Requirement} & \textbf{Priority} \\ \hline
  Receive and forward HTTP 1.X requests & 1\\ \hline
  Allow modifications on the payload & 1 \\ \hline
  Allow configuration of HTTP timeouts & 1 \\ \hline
  Keep HTTP-connection alive & 1 \\ \hline
  Support protocol X and y & 2 \\ \hline
\end{tabular}
\caption{Proxy requirements}
\end{table}


\section{Optimization techniques}
The Web service technology enable interoperability between systems, but also
increase the information overhead, requiring higher data rate demands. Employing
Web Services developed for use in civilian networks directly into a \gls{dil}
environment may not perform satisfactorily. To increase the performance we can
apply optimalization techniques. There are many approaches and optimalization
techniques which can be applied at different levels of the protocol stack. In
the coming sections the different optimization techniques are presented and a
overview is presented in \Cref{table:optimalization-overview}. Another issue
that needs to be addressed is, when we have identified optimization techniques,
where do we place them? In the application itself or in a proxy? This is
discussed in the next section.

\subsection{Where to place the optimalization?}
One approach is to modify the Web service application itself. However, this
would mean that every application that is used in a tactical network would
require modification. This would require a lot of resources and severely limit
the flexibility of using Web services. Another solution is, by using proxies, we
can place the optimization there without altering the Web Services themselves.
The only thing required to do is to setup the application to send and receive
data through the proxy. The proxy will take care of the optimization for
tactical networks. This seems like a more reasonable approach and is explored in
this thesis.


%%%%%%%% TABLE OPTIMALIZATION OVERVIEW %%%%%%%%%%%%%
\begin{table}[h]
\begin{tabularx}{\textwidth}{| X | X |}
\hline
  \textbf{Protocol Stack} & \textbf{optimization possibilities} \\ \hline
  The application & Optimize the application\\ \hline
  Web service messaging: SOAP & Optimize SOAP, e.g XML compression \\ \hline
  HTTP/TCP, UDP or other transport protocols & SOAP is transport agnostic. Other
  protocols can be used. \\ \hline
  IP & NATO NEC feasibility study states that all protocols should be over IP. \\
  \hline
  Lower layers & Not in the scope of this thesis. \\ \hline
\end{tabularx}
\caption{Optimization possibilities.} \label{table:optimalization-overview}
\end{table}


\subsection{Compressing the payload}

The first optimization techniques deals with the optimization of the encoding.
By compressing the Web service payload, we can reduce the amount of data that
need to be sent. This optimization technique addresses one of the many
challenges of tactical networks, namely the bandwidth consumption due to large
message sizes.

\begin{itemize}
\item GZIP

\item EFX(Efficient XML). XML spesefikt, får ikke brukt på REST når vi har andre
payloads..

\end{itemize}

\subsection{Reducing overhead of SOAP}
HTTP/TCP is the most used transport protocol for SOAP messages, but since SOAP is transport protocol agnostic different protocols can be used. Experiments show that this is possible.
