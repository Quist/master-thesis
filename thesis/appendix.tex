\begin{appendices}

\chapter{Proxy Configuration}
\label{appendix-config}

The proxy is configured by passing a path to a valid configuration file at
startup, as seen in \cref{listing:appendix-proxy-start}.  The configuration
fields of the proxy are listed in \cref{appendix-table-config}. When using the
AMQP protocol for inter-proxy communication, a message broker and queue names
must be configured as shown in \cref{appendix-amqp-conf}.

\lstinputlisting[frame=single, language=json, caption="How to start the proxy", label=listing:appendix-proxy-start]{listings/proxy_start.sh}


\begin{table}[h]
\begin{tabularx}{\textwidth}{|l|X|l|l|}
    \hline
    \textbf{Field}         & \textbf{Purpose}                                                   & \textbf{Required} & \textbf{Type} \\ \hline
    useCompression         & Turn GZIP compression on/off                                       & Yes               & boolean       \\ \hline
    protocol               & Inter-proxy communication protocol                                 & Yes               & String        \\ \hline
    hostname               & Hostname to listen on                                              & Yes               & String        \\ \hline
    port                   & Which port the proxy should listen for messages                    & Yes               & Integer       \\ \hline
    targetProxyHostname    & The hostname and the port of the opposite proxy                    & Yes               & String        \\ \hline
    timeout                & The timeout value of a request between proxies                     & No                & Integer       \\ \hline
    useExponentialBackoff  & Turn on/off exponential backoff of retransmission                 & No                & Integer       \\ \hline
    redeliveryDelay        & Number of milliseconds to wait before trying redelivery            & No                & Integer       \\ \hline
    maximumRetransmissions & Maximum number of attempted retransmissions. -1 indicates infinity & No                & Integer       \\ \hline
\end{tabularx}
\caption{Configuration fields of the Proxy}
\label{appendix-table-config}
\end{table}

\lstinputlisting[frame=single, language=json, firstnumber=1, caption="AMQP Proxy configuration example", label=appendix-amqp-conf]{listings/amqp.conf}

\chapter{Network emulating}

\label{appendix-netem-scripts}

This appendix lists the different scripts that were used to emulate the
different types of networks in the evaluation.

\section{\gls{satcom}}
\lstinputlisting[frame=single, language=json, firstnumber=1, caption="Emulating SATCOM"]{../master-test-scripts/satellite.bash}

\section{\gls{los}}
\lstinputlisting[frame=single, language=json, firstnumber=1, caption="Emulating LOS"]{../master-test-scripts/los.bash}

\section{WiFi 1}
\lstinputlisting[frame=single, language=json, firstnumber=1, caption="Emulating WiFi 1"]{../master-test-scripts/wifi1.bash}

\section{WiFi 2}
\lstinputlisting[frame=single, language=json, firstnumber=1, caption="Emulating WiFi 2"]{../master-test-scripts/wifi2.bash}

\section{\gls{cnr}}
\lstinputlisting[frame=single, language=json, firstnumber=1, caption="Emulating CNR"]{../master-test-scripts/cnr.bash}

\section{\gls{edge}}
\lstinputlisting[frame=single, language=json, firstnumber=1, caption="Emulating EDGE"]{../master-test-scripts/edge.bash}



\chapter{Results}
\label{appendix-results}

In this appendix, the data material from the evaluations is presented. Each test
case was run a number of times, ranging from 10 to 100 runs. The mean, \gls{std}
and variance was calculated by using the Apache Commons Mathematics
Library \cite{apache-math-homepage}. An example of how this was done running NFFI
Web service tests can be seen in \cref{listing:statistics}.

\lstinputlisting[frame=single, language=json, firstnumber=1, caption="Calculating statistic values", label=listing:statistics]{listings/stats.java}


We also performed an analysis of the network utilization using Wireshark. This was done by starting a packet capture, running one test run and inspecting the packet capture. The calculation of bytes sent and received was done by:

\begin{enumerate}
    \item Starting Wireshark on the same machine as the client.
    \item Filtering traffic to only show traffic between the IP addresses of the client and Web service.
    \item Using the TCP/UDP conversation view of Wireshark.
\end{enumerate}

\section{Function Tests}

\begin{itemize}
	\item Ping measured to \textasciitilde 1 ms.
	\item Iperf3 measured data rate: 7.76 Mbits/sec.
\end{itemize}


\subsection{NFFI Web Service}

\begin{table}[H]
\begin{tabular}{llllr}
\hline
 Test                   &   Mean &   \gls{std} &   Variance &   Test runs \\
\hline
  Without proxy & 122 ms & 29 & 869 & 300 \\
  Proxy with HTTP & 163 ms & 25 & 601 & 300 \\
  Proxy with HTTP \& GZIP & 99 ms & 19 & 346 & 300 \\
  Proxy with AMQP & 529 ms & 60 & 3690 & 300 \\
  Proxy with AMQP \& GZIP & 490 ms & 62 & 3847 & 300\\
  Proxy with CoAP & 285 ms & 33 & 1122 & 300 \\
  Proxy with CoAP \& GZIP & 122 ms & 33 & 1091 & 300 \\
\end{tabular}
\caption{Mean response times of NFFI Web Service - Function Test}
\end{table}

\begin{table}[H]
\begin{tabular}{lrrrr}
\hline
\multicolumn{1}{l}{}                  & \multicolumn{2}{c}{Client -> Web service}                           & \multicolumn{2}{c}{Web service -> Client}                           \\
\multicolumn{1}{l}{Test} & \multicolumn{1}{l}{Packets sent} & \multicolumn{1}{l}{Bytes sent} & \multicolumn{1}{l}{Packets sent} & \multicolumn{1}{l}{Bytes sent} \\ \hline
Without Proxy                   & 51             & 4609           & 46             & 51706          \\
Proxy with HTTP                 & 45             & 5392           & 44             & 55489          \\
Proxy with HTTP \& GZIP         & 13             & 2781           & 13             & 2585           \\
Proxy with AMQP                 & 73             & 10284          & 94             & 64472          \\
Proxy with AMQP \& GZIP         & 57             & 8731           & 62             & 15244          \\
Proxy with CoAP                 & 101            & 8120           & 101            & 57137          \\
Proxy with CoAP \& GZIP         & 11             & 1680           & 11             & 5502           \\
\end{tabular}

\caption{Wireshark analysis of NFFI Web Service - Packets \& Bytes Sent -- Function Test}
\end{table}

\subsection{RESTful Car System}

\begin{table}[H]
\begin{tabular}{lrrrr}
\hline
 Test                   &   Mean &   Std. Deviation &   Variance &   Test runs \\
\hline
 Without proxy          &     30 &               12 &        147 &         100 \\
 Proxy with HTTP        &    160 &               97 &       9486 &         100 \\
 Proxy with HTTP \& GZIP &    159 &               76 &       5822 &         100 \\
 Proxy with AMQP        &   1919 &              128 &      16388 &         100 \\
 Proxy with AMQP \& GZIP &   1880 &              109 &      11919 &         100 \\
 Proxy with CoAP        &    124 &               64 &       4079 &         100 \\
 Proxy with CoAP \& GZIP &    128 &               64 &       4109 &         100 \\
\hline
\end{tabular}
\caption{Mean response times of RESTful Car System - Function Test}
\end{table}

\begin{table}[H]
\begin{tabular}{lrrrr}
\hline
\multicolumn{1}{l}{}                  & \multicolumn{2}{c}{Client -> Web service}                           & \multicolumn{2}{c}{Web service -> Client}                           \\
\multicolumn{1}{l}{Test} & \multicolumn{1}{l}{Packets sent} & \multicolumn{1}{l}{Bytes sent} & \multicolumn{1}{l}{Packets sent} & \multicolumn{1}{l}{Bytes sent} \\ \hline
Without Proxy                   & 25             & 4738           & 21             & 5638           \\
Proxy with HTTP                 & 28             & 9677           & 26             & 15147          \\
Proxy with HTTP \& GZIP         & 28             & 8735           & 28             & 12993          \\
Proxy with AMQP                 & 180            & 30366          & 203            & 47484          \\
Proxy with AMQP \& GZIP         & 190            & 30224          & 207            & 42314          \\
Proxy with CoAP                 & 12             & 4757           & 12             & 8369           \\
Proxy with CoAP \& GZIP         & 12             & 3943           & 12             & 6053           \\
\end{tabular}

\caption{Wireshark analysis of RESTful Car System - Packets \& Bytes Sent - Function Test}
\end{table}

\section{Satellite Tests}

\begin{itemize}
	\item Ping measured to \textasciitilde 1100 ms.
	\item Iperf3 measured data rate: 402/291 Kbits/sec.
\end{itemize}

\subsection{NFFI Web Service}

\begin{table}[H]
\begin{tabular}{llrrr}
\hline
 Test                   & Mean      &   Std. Deviation &   Variance &   Test runs \\
\hline
 Without proxy          & 4978 ms   &              378 &     142762 &          10 \\
 Proxy with HTTP        & 4511 ms   &               71 &       5009 &          10 \\
 Proxy with HTTP \& GZIP & 3530 ms   &               50 &       2472 &          10 \\
 Proxy with AMQP        & 25709 ms  &              793 &     628112 &          10 \\
 Proxy with AMQP \& GZIP & 25780 ms  &             1159 &    1343947 &          10 \\
 Proxy with CoAP        & 111636 ms &               59 &       3437 &          10 \\
 Proxy with CoAP \& GZIP & 12347 ms  &               41 &       1652 &          10 \\
\hline
\end{tabular}
\caption{Mean response times of NFFI Web Service - Satellite test}
\end{table}

\begin{table}[H]
\begin{tabularx}{\textwidth}{lXXXX}
\hline
\multicolumn{1}{l}{}                  & \multicolumn{2}{c}{Client->Web service}                           & \multicolumn{2}{c}{Web service->Client}                           \\
\multicolumn{1}{l}{Test} & \multicolumn{1}{l}{P. sent} & \multicolumn{1}{l}{B. sent} & \multicolumn{1}{l}{P.sent} & \multicolumn{1}{l}{B.sent} \\ \hline
Without Proxy                   & 54             & 4811           & 47             & 51623          \\
Proxy with HTTP                 & 47             & 5532           & 45             & 55563          \\
Proxy with HTTP \& GZIP         & 16             & 2987           & 14             & 7177           \\
Proxy with AMQP                 & 88             & 11342          & 102            & 65040          \\
Proxy with AMQP \& GZIP         & 71             & 9731           & 68             & 15679          \\
Proxy with CoAP                 & 101            & 7810           & 101            & 56827          \\
Proxy with CoAP \& GZIP         & 11             & 1668           & 11             & 5486           \\
\end{tabularx}

\caption{Wireshark analysis of NFFI Web Service - Packets \& Bytes Sent - Satellite test}
\end{table}

\subsection{RESTful Car System}

\begin{table}[H]
\begin{tabular}{llrrr}
\hline
 Test                   & Mean      &   Std. Deviation &   Variance &   Test runs \\
\hline
 Without proxy          & 13386 ms  &              401 &     160523 &          10 \\
 Proxy with HTTP        & 13643 ms  &              427 &     182464 &          10 \\
 Proxy with HTTP \& GZIP & 13825 ms  &              897 &     804893 &          10 \\
 Proxy with AMQP        & 102748 ms &             3065 &    9396423 &          10 \\
 Proxy with AMQP \& GZIP & 94163 ms  &              568 &     322659 &          10 \\
 Proxy with CoAP        & 13545 ms  &              217 &      47260 &          10 \\
 Proxy with CoAP \& GZIP & 13562 ms  &              223 &      49522 &          10 \\
\hline
\end{tabular}
\caption{Mean response times of RESTful Car System - Satellite test}
\end{table}

\begin{table}[H]
\begin{tabularx}{\textwidth}{lXXXX}
\hline
\multicolumn{1}{l}{}                  & \multicolumn{2}{c}{Client->Web service}                           & \multicolumn{2}{c}{Web service->Client}                           \\
\multicolumn{1}{l}{Test} & \multicolumn{1}{l}{P. sent} & \multicolumn{1}{l}{B. sent} & \multicolumn{1}{l}{P.sent} & \multicolumn{1}{l}{B.sent} \\ \hline
Without Proxy                   & 27             & 4878           & 22             & 5712           \\
Proxy with HTTP                 & 26             & 9538           & 25             & 15075          \\
Proxy with HTTP \& GZIP         & 30             & 8873           & 28             & 13010          \\
Proxy with AMQP                 & 244            & 34841          & 238            & 49914          \\
Proxy with AMQP \& GZIP         & 240            & 33739          & 240            & 44625          \\
Proxy with CoAP                 & 12             & 4751           & 12             & 8380           \\
Proxy with CoAP \& GZIP         & 12             & 3940           & 12             & 6063           \\
\end{tabularx}

\caption{Wireshark analysis of RESTful Car System - Packets \& Bytes Sent - Satellite test}
\end{table}


\section{Line-of-Sight Tests}

\begin{itemize}
	\item Ping measured to \textasciitilde 11 ms.
	\item Iperf3 measured data rate: 2.34/2.15 Mbits/sec.
\end{itemize}

\subsection{NFFI Web Service}

\begin{table}[H]
\begin{tabular}{llllr}
\hline
 Test                   &   Mean &   STD &   Variance &   Test runs \\
\hline
  Without proxy & 242 ms & 26 & 663 & 100 \\
  Proxy with HTTP & 299 ms & 40 & 1577 & 100 \\
  Proxy with HTTP \& GZIP & 162 ms & 34 & 1177 & 100 \\
  Proxy with AMQP & 821 ms & 60 & 3588 & 100 \\
  Proxy with AMQP \& GZIP & 693 ms & 75 & 5632 & 100\\
  Proxy with CoAP & 1359 ms & 45 & 1988 & 100 \\
  Proxy with CoAP \& GZIP & 262 ms & 36 & 1314 & 100 \\
\end{tabular}
\caption{Mean response times of NFFI Web Service - LOS test}
\end{table}

\begin{table}[H]
\begin{tabularx}{\textwidth}{lXXXX}
\hline
\multicolumn{1}{l}{}                  & \multicolumn{2}{c}{Client->Web service}                           & \multicolumn{2}{c}{Web service->Client}                           \\
\multicolumn{1}{l}{Test} & \multicolumn{1}{l}{P. sent} & \multicolumn{1}{l}{B. sent} & \multicolumn{1}{l}{P.sent} & \multicolumn{1}{l}{B.sent} \\ \hline
Without Proxy                   & 46             & 4267           & 43             & 51343          \\
Proxy with HTTP                 & 43             & 5260           & 44             & 55489          \\
Proxy with HTTP \& GZIP         & 14             & 2847           & 13             & 7103           \\
Proxy with AMQP                 & 68             & 9950           & 91             & 64274          \\
Proxy with AMQP \& GZIP         & 54             & 8529           & 59             & 15044          \\
Proxy with CoAP                 & 101            & 7565           & 101            & 56582          \\
Proxy with CoAP \& GZIP         & 11             & 1647           & 11             & 5466           \\
\end{tabularx}

\caption{Wireshark analysis of NFFI Web Service - Packets \& Bytes Sent - LOS test}
\end{table}


\subsection{RESTful Car System}

\begin{table}[H]
\begin{tabularx}{\textwidth}{llrrr}
\hline
 Test                   & Mean    &   Std. Deviation &   Variance &   Test runs \\
\hline
 Without proxy          & 156 ms  &               15 &        214 &         100 \\
 Proxy with HTTP        & 288 ms  &               77 &       6000 &         100 \\
 Proxy with HTTP \& GZIP & 292 ms  &               86 &       7382 &         100 \\
 Proxy with AMQP        & 2567 ms &              102 &      10333 &         100 \\
 Proxy with AMQP \& GZIP & 2579 ms &              129 &      16595 &         100 \\
 Proxy with CoAP        & 256 ms  &               69 &       4775 &         100 \\
 Proxy with CoAP \& GZIP & 263 ms  &               69 &       4693 &         100 \\
\hline
\end{tabularx}
\caption{Mean response times of RESTful Car System - LOS test}
\end{table}

\begin{table}[H]
\begin{tabularx}{\textwidth}{lXXXX}
\hline
\multicolumn{1}{l}{}                  & \multicolumn{2}{c}{Client->Web service}                           & \multicolumn{2}{c}{Web service->Client}                           \\
\multicolumn{1}{l}{Test} & \multicolumn{1}{l}{P. sent} & \multicolumn{1}{l}{B. sent} & \multicolumn{1}{l}{P.sent} & \multicolumn{1}{l}{B.sent} \\ \hline
Without Proxy                   & 25             & 4738           & 21             & 5638           \\
Proxy with HTTP                 & 28             & 9704           & 26             & 15201          \\
Proxy with HTTP \& GZIP         & 24             & 8486           & 24             & 8486           \\
Proxy with AMQP                 & 189            & 30968          & 201            & 47352          \\
Proxy with AMQP \& GZIP         & 187            & 29979          & 201            & 41927          \\
Proxy with CoAP                 & 12             & 4756           & 12             & 8397           \\
Proxy with CoAP \& GZIP         & 12             & 3934           & 12             & 6059           \\
\end{tabularx}

\caption{Wireshark analysis of RESTful Car System - Packets \& Bytes Sent - LOS test}
\end{table}


\section{WiFi 1 tests}

\begin{itemize}
	\item Ping measured to \textasciitilde 200 ms.
	\item Iperf3 measured data rate: 1.72/1.67 Mbits/sec.
\end{itemize}


\subsection{NFFI Web Service}

\begin{table}[H]
\begin{tabular}{llllr}
\hline
 Test                   &   Mean &   STD &   Variance &   Test runs \\
\hline
  Without proxy & 1202 ms & 162 & 26326 & 100 \\
  Proxy with HTTP & 1213 ms & 354 & 125628 & 100 \\
  Proxy with HTTP \& GZIP & 820 ms & 154 & 23586 & 100 \\
  Proxy with AMQP & 5026 ms & 460 & 211385 & 100 \\
  Proxy with AMQP \& GZIP & 4964 ms & 637 & 405390 & 100\\
  Proxy with CoAP & 25615 ms & 3185 & 10142866 & 10 \\
  Proxy with CoAP \& GZIP & 2823 ms & 1425 & 2031770 & 100 \\
\end{tabular}
\caption{Mean response times of NFFI Web Service - WiFi 1 test}
\end{table}

\begin{table}[H]
\begin{tabularx}{\textwidth}{lXXXX}
\hline
\multicolumn{1}{l}{}                  & \multicolumn{2}{c}{Client->Web service}                           & \multicolumn{2}{c}{Web service->Client}                           \\
\multicolumn{1}{l}{Test} & \multicolumn{1}{l}{P. sent} & \multicolumn{1}{l}{B. sent} & \multicolumn{1}{l}{P.sent} & \multicolumn{1}{l}{B.sent} \\ \hline
Without Proxy                   & 50             & 4531           & 45             & 51475          \\
Proxy with HTTP                 & 45             & 5560           & 45             & 57003          \\
Proxy with HTTP \& GZIP         & 13             & 2793           & 14             & 8297           \\
Proxy with AMQP                 & 76             & 10494          & 93             & 64406          \\
Proxy with AMQP \& GZIP         & 60             & 8941           & 60             & 15126          \\
Proxy with CoAP                 & 104            & 9214           & 104            & 58817          \\
Proxy with CoAP \& GZIP         & 11             & 1682           & 11             & 5491           \\
\end{tabularx}

\caption{Wireshark analysis of NFFI Web Service - Packets \& Bytes Sent - WiFi 1 test}
\end{table}

\subsection{RESTful Car System}

\begin{table}[H]
\begin{tabularx}{\textwidth}{llrrr}
\hline
 Test                   & Mean     &   Std. Deviation &   Variance &   Test runs \\
\hline
 Without proxy          & 2581 ms  &              265 &      70406 &         100 \\
 Proxy with HTTP        & 2728 ms  &              270 &      73000 &         100 \\
 Proxy with HTTP \& GZIP & 2818 ms  &              369 &     136307 &         100 \\
 Proxy with AMQP        & 19236 ms &              490 &     240174 &          10 \\
 Proxy with AMQP \& GZIP & 18925 ms &              722 &     521008 &          10 \\
 Proxy with CoAP        & 3184 ms  &             1565 &    2447810 &         100 \\
 Proxy with CoAP \& GZIP & 3024 ms  &              946 &     894686 &         100 \\
\hline
\end{tabularx}
\caption{Mean response times of RESTful Car System - WiFi 1 test}
\end{table}

\begin{table}[H]
\begin{tabular}{lrrrr}
\hline
\multicolumn{1}{l}{}                  & \multicolumn{2}{c}{Client -> Web service}                           & \multicolumn{2}{c}{Web service -> Client}                           \\
\multicolumn{1}{l}{Test} & \multicolumn{1}{l}{Packets sent} & \multicolumn{1}{l}{Bytes sent} & \multicolumn{1}{l}{Packets sent} & \multicolumn{1}{l}{Bytes sent} \\ \hline
Without Proxy                   & 28             & 5146           & 22             & 6060           \\
Proxy with HTTP                 & 26             & 9564           & 24             & 15061          \\
Proxy with HTTP \& GZIP         & 30             & 9476           & 27             & 12925          \\
Proxy with AMQP                 & 192            & 31450          & 211            & 49663          \\
Proxy with AMQP \& GZIP         & 198            & 30730          & 208            & 42380          \\
Proxy with CoAP                 & 12             & 4754           & 12             & 8366           \\
Proxy with CoAP \& GZIP         & 12             & 3945           & 12             & 6035           \\
\end{tabular}
\caption{Wireshark analysis of RESTful Car System - Packets \& Bytes Sent - WiFi 1 test}
\end{table}


\section{WiFI 2 Tests}

\begin{itemize}
	\item Ping measured to \textasciitilde 200 ms.
	\item Iperf3 measured data rate: 125/99.6 Kbits/sec.
\end{itemize}

\subsection{NFFI Web Service}

\begin{table}[H]
\begin{tabularx}{\textwidth}{llllr}
\hline
 Test                   & Mean     & STD   & Variance   &   Test runs \\
\hline
 Without proxy          & 13235 ms & 9070  & 82266227   &          10 \\
 Proxy with HTTP        & 12042 ms & 6908  & 47717943   &          10 \\
 Proxy with HTTP \& GZIP & 3938 ms  & 4793  & 22970668   &          20 \\
 Proxy with AMQP        & 31096 ms & 20578 & 423443967  &          10 \\
 Proxy with AMQP \& GZIP & 15243 ms & 9267  & 85874508   &          10 \\
 Proxy with CoAP        & 0 ms     & -     & -          &           1 \\
 Proxy with CoAP \& GZIP & 37073 ms & 46459 & 2158462617 &          20 \\
\hline
\end{tabularx}
\caption{Mean response times of NFFI Web Service - WiFi 2 test}
\end{table}

\begin{table}[H]
\begin{tabularx}{\textwidth}{lXXXX}
\hline
\multicolumn{1}{l}{}                  & \multicolumn{2}{c}{Client->Web service}                           & \multicolumn{2}{c}{Web service->Client}                           \\
\multicolumn{1}{l}{Test} & \multicolumn{1}{l}{P. sent} & \multicolumn{1}{l}{B. sent} & \multicolumn{1}{l}{P.sent} & \multicolumn{1}{l}{B.sent} \\ \hline
Without Proxy                   & 51             & 5198           & 54             & 64805          \\
Proxy with HTTP                 & 45             & 5736           & 52             & 67172          \\
Proxy with HTTP \& GZIP         & 15             & 3846           & 13             & 8548           \\
Proxy with AMQP                 & 101            & 12862          & 111            & 78455          \\
Proxy with AMQP \& GZIP         & 76             & 10653          & 71             & 16773          \\
Proxy with CoAP                 & 0              & 0              & 0              & 0              \\
Proxy with CoAP \& GZIP         & 14             & 1863           & 12             & 6061           \\
\end{tabularx}

\caption{Wireshark analysis of NFFI Web Service - Packets \& Bytes Sent - WiFi 2 test}
\end{table}

\subsection{RESTful Car System}

\begin{table}[H]
\begin{tabular}{llrrr}
\hline
 Test                   & Mean     &   Std. Deviation &   Variance &   Test runs \\
\hline
 Without proxy          & 8132 ms  &             7853 &   61661813 &          20 \\
 Proxy with HTTP        & 7259 ms  &             1764 &    3111671 &          20 \\
 Proxy with HTTP \& GZIP & 8611 ms  &             2815 &    7924419 &          20 \\
 Proxy with AMQP        & 85609 ms &            26355 &  694606921 &          10 \\
 Proxy with AMQP \& GZIP & 76636 ms &            34666 & 1201698634 &          10 \\
 Proxy with CoAP        & 24183 ms &            14067 &  197893185 &          10 \\
 Proxy with CoAP \& GZIP & 21096 ms &            11300 &  127698638 &          10 \\
\hline
\end{tabular}
\caption{Mean response times of RESTful Car System - WiFi 2 test}
\end{table}

\begin{table}[H]
\begin{tabularx}{\textwidth}{lXXXX}
\hline
\multicolumn{1}{l}{}                  & \multicolumn{2}{c}{Client->Web service}                           & \multicolumn{2}{c}{Web service->Client}                           \\
\multicolumn{1}{l}{Test} & \multicolumn{1}{l}{P. sent} & \multicolumn{1}{l}{B. sent} & \multicolumn{1}{l}{P.sent} & \multicolumn{1}{l}{B.sent} \\ \hline
Without proxy                           & 32                                            & 6 136                                    & 39                                            & 11 065                                   \\
Proxy with HTTP                         & 37                                            & 12 434                                   & 30                                            & 16 596                                   \\
Proxy with HTTP \& GZIP                 & 31                                            & 9 575                                    & 28                                            & 13 901 \\
Proxy with AMQP                         & 332                                            & 49 793                                    & 317                                            & 65 154 \\
Proxy with AMQP \& GZIP                 & 231                                            & 34 501                                    & 243                                            & 54 626 \\
Proxy with CoAP                         & 18                                            & 6 895                                    & 15                                            & 10 640 \\
Proxy with CoAP \& GZIP                 & 24                                            & 7 730                                    & 17                                            & 8 566

\end{tabularx}
\caption{Wireshark analysis of RESTful Car System - Packets \& Bytes Sent - WiFi 2 test}

\end{table}


\section{Combat Net Radio Tests}

\begin{itemize}
	\item Ping measured to \textasciitilde 200 ms.
	\item Iperf3 measured data rate: 41/36 Kbits/sec.
\end{itemize}


\subsection{NFFI Web service}

\begin{table}[H]
\begin{tabular}{llllr}
\hline
 Test                   & Mean    & Std. Deviation   & Variance   &   Test runs \\
\hline
 Without proxy          & 44332   & 773              & 597167     &          10 \\
 Proxy with HTTP        & 48434   & 3255             & 10595445   &          10 \\
 Proxy with HTTP \& GZIP & 5696    & 522              & 272157     &          10 \\
 Proxy with AMQP        & Timeout & -                & -          &           1 \\
 Proxy with AMQP \& GZIP & 13241   & 1071             & 1147182    &          10 \\
 Proxy with CoAP        & 48302   & 1046             & 1095139    &          10 \\
 Proxy with CoAP \& GZIP & 3803    & 1218             & 1482324    &          10 \\
\hline
\end{tabular}
\caption{Mean response times of NFFI Web Service - CNR test}
\end{table}

\begin{table}[H]
\begin{tabularx}{\textwidth}{lXXXX}
\hline
\multicolumn{1}{l}{}                  & \multicolumn{2}{c}{Client->Web service}                           & \multicolumn{2}{c}{Web service->Client}                           \\
\multicolumn{1}{l}{Test} & \multicolumn{1}{l}{P. sent} & \multicolumn{1}{l}{B. sent} & \multicolumn{1}{l}{P.sent} & \multicolumn{1}{l}{B.sent} \\ \hline
Without Proxy                   & 70             & 5831           & 71             & 64836          \\
Proxy with HTTP                 & 66             & 6670           & 67             & 71551          \\
Proxy with HTTP \& GZIP         & 14             & 2847           & 13             & 7104           \\
Proxy with AMQP                 & 0              & 0              & 0              & 0              \\
Proxy with AMQP \& GZIP         & 56             & 8697           & 62             & 15253          \\
Proxy with CoAP                 & 103            & 7718           & 103            & 57745          \\
Proxy with CoAP \& GZIP         & 11             & 1652           & 11             & 5741           \\
\end{tabularx}

\caption{Wireshark analysis of NFFI Web Service - Packets \& Bytes Sent - CNR test}
\end{table}


\subsection{RESTful Car System}

\begin{table}[H]
\begin{tabular}{lrrrr}
\hline
 Test                   &   Mean &   Std. Deviation &   Variance &   Test runs \\
\hline
 Without proxy          &   4055 &              960 &     921629 &          20 \\
 Proxy with HTTP        &  11478 &             2842 &    8077362 &          10 \\
 Proxy with HTTP \& GZIP &   9526 &             2701 &    7292955 &          10 \\
 Proxy with AMQP        &  41255 &             3171 &          1 &          10 \\
 Proxy with AMQP \& GZIP &  36540 &             3281 &   10767443 &          10 \\
 Proxy with CoAP        &   5872 &             2056 &    4226612 &          10 \\
 Proxy with CoAP \& GZIP &   3840 &             1366 &    1865202 &          10 \\
\hline
\end{tabular}
\caption{Mean response times of RESTful Car System - CNR test}
\end{table}

\begin{table}[H]
\begin{tabularx}{\textwidth}{lXXXX}
\hline
\multicolumn{1}{l}{}                  & \multicolumn{2}{c}{Client->Web service}                           & \multicolumn{2}{c}{Web service->Client}                           \\
\multicolumn{1}{l}{Test} & \multicolumn{1}{l}{P. sent} & \multicolumn{1}{l}{B. sent} & \multicolumn{1}{l}{P.sent} & \multicolumn{1}{l}{B.sent} \\ \hline
Without Proxy                   & 25             & 4738           & 21             & 5638           \\
Proxy with HTTP                 & 28             & 9677           & 27             & 15213          \\
Proxy with HTTP \& GZIP         & 24             & 8473           & 24             & 12762          \\
Proxy with AMQP                 & 233            & 34279          & 240            & 54257          \\
Proxy with AMQP \& GZIP         & 220            & 32420          & 225            & 45833          \\
Proxy with CoAP                 & 14             & 5435           & 13             & 9065           \\
Proxy with CoAP \& GZIP         & 12             & 3919           & 12             & 6023           \\
\end{tabularx}

\caption{Wireshark analysis of RESTful Car System - Packets \& Bytes Sent - CNR test}
\end{table}


\section{EDGE Tests}

\begin{itemize}
	\item Ping measured to \textasciitilde 400 ms.
	\item Iperf3 measured data rate: 140/97 Kbits/sec.
\end{itemize}

\subsection{NFFI Web service}

\begin{table}[H]
\begin{tabularx}{\textwidth}{llrrr}
\hline
 Test                   & Mean     &   Std. Deviation &   Variance &   Test runs \\
\hline
 Without proxy          & 2437 ms  &               18 &        340 &          20 \\
 Proxy with HTTP        & 2587 ms  &               40 &       1583 &          20 \\
 Proxy with HTTP \& GZIP & 1381 ms  &               38 &       1477 &          20 \\
 Proxy with AMQP        & 9334 ms  &               65 &       4216 &          20 \\
 Proxy with AMQP \& GZIP & 8909 ms  &              158 &      24930 &          20 \\
 Proxy with CoAP        & 40855 ms &               46 &       2151 &          20 \\
 Proxy with CoAP \& GZIP & 4630 ms  &               38 &       1481 &          20 \\
\hline
\end{tabularx}
\caption{Mean response times of NFFI Web Service - EDGE test}
\end{table}

\begin{table}[H]
\begin{tabular}{lrrrr}
\hline
\multicolumn{1}{l}{}                  & \multicolumn{2}{c}{Client -> Web service}                           & \multicolumn{2}{c}{Web service -> Client}                           \\
\multicolumn{1}{l}{Test} & \multicolumn{1}{l}{Packets sent} & \multicolumn{1}{l}{Bytes sent} & \multicolumn{1}{l}{Packets sent} & \multicolumn{1}{l}{Bytes sent} \\ \hline
Without Proxy                   & 50             & 4531           & 45             & 51475          \\
Proxy with HTTP                 & 45             & 5404           & 44             & 55489          \\
Proxy with HTTP \& GZIP         & 14             & 2847           & 13             & 7101           \\
Proxy with AMQP                 & 78             & 10630          & 95             & 64538          \\
Proxy with AMQP \& GZIP         & 59             & 8871           & 59             & 15050          \\
Proxy with CoAP                 & 101            & 7948           & 101            & 56965          \\
Proxy with CoAP \& GZIP         & 11             & 1660           & 11             & 5480           \\
\end{tabular}
\caption{Wireshark analysis of NFFI Web Service - Packets \& Bytes Sent - EDGE test}
\end{table}

\subsection{RESTful Car System}

\begin{table}[H]
\begin{tabularx}{\textwidth}{llrrr}
\hline
 Test                   & Mean     &   Std. Deviation &   Variance &   Test runs \\
\hline
 Without proxy          & 4884 ms  &              132 &      17328 &          20 \\
 Proxy with HTTP        & 5116 ms  &              139 &      19459 &          20 \\
 Proxy with HTTP \& GZIP & 5061 ms  &              138 &      18960 &          20 \\
 Proxy with AMQP        & 35393 ms &              764 &     583712 &          20 \\
 Proxy with AMQP \& GZIP & 35192 ms &              446 &     199015 &          20 \\
 Proxy with CoAP        & 5063 ms  &               59 &       3488 &          20 \\
 Proxy with CoAP \& GZIP & 5064 ms  &               60 &       3604 &          20 \\
\hline
\end{tabularx}
\caption{Mean response times of RESTful Car System - EDGE test}
\end{table}

\begin{table}[H]
\begin{tabularx}{\textwidth}{lXXXX}
\hline
\multicolumn{1}{l}{}                  & \multicolumn{2}{c}{Client->Web service}                           & \multicolumn{2}{c}{Web service->Client}                           \\
\multicolumn{1}{l}{Test} & \multicolumn{1}{l}{P. sent} & \multicolumn{1}{l}{B. sent} & \multicolumn{1}{l}{P.sent} & \multicolumn{1}{l}{B.sent} \\ \hline
Without Proxy                   & 27             & 4886           & 23             & 5570           \\
Proxy with HTTP                 & 28             & 9677           & 27             & 15213          \\
Proxy with HTTP \& GZIP         & 29             & 8798           & 27             & 12941          \\
Proxy with AMQP                 & 194            & 31292          & 201            & 47325          \\
Proxy with AMQP \& GZIP         & 201            & 31006          & 212            & 42611          \\
Proxy with CoAP                 & 12             & 4761           & 12             & 8375           \\
Proxy with CoAP \& GZIP         & 12             & 3943           & 12             & 6068           \\
\end{tabularx}

\caption{Wireshark analysis of RESTful Car System - Packets \& Bytes Sent - EDGE test}
\end{table}


\section{Tactical Broadband Tests}

\begin{itemize}
	\item Ping measured to \textasciitilde 23 ms.
	\item Iperf3 measured data rate: 99/82 Kbits/sec.
\end{itemize}

\subsection{NFFI Web service}

\begin{table}[H]
\begin{tabular}{llllr}
\hline
 Test                   &   Mean &   STD &   Variance &   Test runs \\
\hline
  Without proxy & 1379 ms & 230 & 52988 & 100 \\
  Proxy with HTTP & 1313 ms & 139 & 19430 & 100 \\
  Proxy with HTTP \& GZIP & 464 ms & 77 & 5874 & 100 \\
  Proxy with AMQP & 2838 ms & 318 & 101162 & 100 \\
  Proxy with AMQP \& GZIP & 1841 ms & 220 & 48240 & 100\\
  Proxy with CoAP & 2720 ms & 120 & 14457 & 100 \\
  Proxy with CoAP \& GZIP & 463 ms & 25 & 618 & 100 \\
\end{tabular}
\caption{Mean response times of NFFI Web Service - Tactical Broadband test}
\end{table}

\subsection{RESTful Car System}

\begin{table}[H]
\begin{tabular}{llllr}
\hline
 Test                   &   Mean &   STD  &   Variance &   Test runs \\
\hline
  Without proxy & 1061 ms & X & X & 100 \\
  Proxy with HTTP & 1522 ms & X & X & 100 \\
  Proxy with HTTP \& GZIP & 1404 ms & X & X & 100 \\
  Proxy with AMQP & 7353 ms & X & X & 100 \\
  Proxy with AMQP \& GZIP & 7241 ms & X & X & 100\\
  Proxy with CoAP & 906 ms & X & X & 100 \\
  Proxy with CoAP \& GZIP & 840 ms & X & X & 100 \\
\end{tabular}
\caption{Mean response times of RESTful Car System - Tactical Broadband test}
\end{table}

\chapter{Source Code}
\label{appendix-source-code}

The source code of the different projects are available at the URL provided in
\cref{ab}.

\begin{table}[H]
\centering
\begin{tabular}{|l|l|}
\hline
\textbf{Project}     & \textbf{URL}                                 \\ \hline
Proxy                & https://github.com/Quist/dil-proxy           \\ \hline
Camel CoAP Component & https://github.com/Quist/camel-coap          \\ \hline
Car System Service   & https://github.com/Quist/master-car-backend  \\ \hline
Car System Client    & https://github.com/Quist/master-car-client   \\ \hline
NFFI Service         & https://github.com/Quist/master-nffi-service \\ \hline
NFFI Client          & https://github.com/Quist/master-nffi-client  \\ \hline
\end{tabular}
\caption{Source code repositories}
\label{ab}
\end{table}


\end{appendices}
